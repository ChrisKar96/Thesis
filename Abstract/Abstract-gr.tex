\chapter*{Περίληψη}
Η δικτυακή εκκίνηση, εν συντομία netboot, είναι η διαδικασία εκκίνησης ενός υπολογιστή από ένα δίκτυο σε αντίθεση με μια τοπική συσκευή αποθήκευσης. Αυτή η τεχνική χρησιμοποιείται συχνά από δρομολογητές, σταθμούς εργασίας χωρίς δίσκο και υπολογιστές με κεντρική διαχείριση, όπως αυτοί που βρίσκονται σε επιχειρήσεις, δημόσιες βιβλιοθήκες και εκπαιδευτικά ιδρύματα. Η εκκίνηση μέσω δικτύου επιτρέπει την κεντρική διαχείριση της αποθήκευσης δίσκων, η οποία μπορεί δυνητικά να οδηγήσει σε μείωση του κόστους κεφαλαίου και συντήρησης. Μπορεί επίσης να είναι χρήσιμη σε υπολογιστικές συστάδες, όπου οι κόμβοι μπορεί να μην διαθέτουν τοπικούς δίσκους.

Σκοπός της παρούσας διπλωματικής εργασίας είναι να βοηθήσει τους διαχειριστές εργαστηρίου στο Εργαστήριο Ρομποτικής, Ενσωματωμένων και Ολοκληρωμένων Συστημάτων του Πανεπιστημίου Δυτικής Μακεδονίας στη διαχείριση των διαθέσιμων μπλοκ iPXE και των μενού εκκίνησης, καθώς και του χρονοδιαγράμματος που επιτρέπεται η δικτυακή εκκίνηση των υπολογιστών του εργαστηρίου.

Η διπλωματική εργασία περιγράφει το σχεδιασμό και την ανάπτυξη μιας διαδικτυακής πλατφόρμας διαχείρισης για την δικτυακή εκκίνηση υπολογιστών. Στόχος της πλατφόρμας είναι να βοηθήσει τους διαχειριστές στην ομαδοποίηση των υπολογιστών του εργαστηρίου και στην αποτελεσματική διαχείριση του χρονοδιαγράμματος εκκίνησης κάθε ομάδας. Οι διαχειριστές δημιουργούν καταχωρήσεις μενού εκκίνησης στην πλατφόρμα. Αυτές οι καταχωρήσεις αποτελούνται από επαναχρησιμοποιήσιμα μπλοκ iPXE που οι ομάδες υπολογιστών μπορούν να παρέχουν στα μέλη τους κατά την εκκίνηση. Τελικά, σε κάθε υπολογιστή που εκκινείται παρέχεται ένα δυναμικά παραγόμενο μενού εκκίνησης που είναι προσαρμοσμένο στο ενεργό πρόγραμμα για την ομάδα της οποίας μέλος είναι ο υπολογιστής και της τρέχουσας ημερομηνίας και ώρας. Τέλος, η πλατφόρμα παρέχει αρχεία καταγραφής συμβάντων και διευκολύνει τους διαχειριστές στην παρακολούθηση των ενεργών συστημάτων υπολογιστών στο εργαστήριο.

Η πλατφόρμα δημιουργήθηκε χρησιμοποιώντας τεχνολογία λογισμικού αιχμής, καινοτόμες μεθόδους προγραμματισμού και γλώσσες προγραμματισμού ανοικτού κώδικα. Για τον προγραμματισμό του back-end της πλατφόρμας και του REST API χρησιμοποιήθηκε PHP με την υποστήριξη του CodeIgniter 4 Framework. Το front-end, από την άλλη πλευρά, χρησιμοποίησε HTML, CSS, JavaScript και Bootstrap. Όσον αφορά τις τεχνικές προγραμματισμού που χρησιμοποιήθηκαν, το back-end αναπτύχθηκε ακολουθώντας τα πρότυπα αντικειμενοστραφούς προγραμματισμού του σχεδιασμού MVC. Το front-end, με τη σειρά του, ακολούθησε τις αρχές του Responsive design. Η πλατφόρμα αποθηκεύει τα δεδομένα της σε μια σχεσιακή βάση δεδομένων MySQL και οι διεπαφές του REST API τεκμηριώθηκαν και απεικονίστηκαν με τη χρήση του OpenAPI Swagger. Τα εργαλεία ανάπτυξης που χρησιμοποιήθηκαν ήταν το JetBrains PHPStorm και το Axosoft GitKraken.

% put this on the bottom
\vfill
\textbf{Λέξεις κλειδιά}: DHCP, BOOTP, TFTP, IPXE, NFS, HTTPFS2 / FTPS, ISCSI, PHP, CODEIGNITER, MYSQL, BOOTSTRAP