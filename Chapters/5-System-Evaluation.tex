\chapter{Αξιολόγηση Συστήματος}
Σε αυτό το κεφάλαιο, αναλύονται λεπτομερώς οι δοκιμές που έγιναν για την εξακρίβωση της ορθής λειτουργίας της δικτυακής εφαρμογής. Επίσης, καταγράφεται το πλάνο ελέγχου ορθής λειτουργίας της πλατφόρμας και γίνεται μελέτη κλιμάκωσης του συστήματος.

\section{Δοκιμή Πλατφόρμας}
\subsection{Οι δοκιμές στην ανάπτυξη λογισμικού}
Η δοκιμή λογισμικού είναι η διαδικασία αντικειμενικής αξιολόγησης και επαλήθευσης της λειτουργικότητας μιας εφαρμογής λογισμικού για τον εντοπισμό πιθανών σφαλμάτων ή κακής συμπεριφοράς. Προσδιορίζει αν το λογισμικό που αναπτύχθηκε ευθυγραμμίζεται με τις καθορισμένες απαιτήσεις και εντοπίζει τυχόν ελαττώματα στο λογισμικό, ώστε να διασφαλίζεται η παραγωγή ενός προϊόντος υψηλής ποιότητας. Εντοπίζει τυχόν σφάλματα, κενά, παραλείψεις ή ελλείψεις απαιτήσεων που δεν ευθυγραμμίζονται με τις απαραίτητες.

Η δοκιμή λογισμικού περιλαμβάνει την εκτέλεση ενός συστατικού λογισμικού ή συστήματος για την αξιολόγηση μιας ή περισσότερων ιδιοτήτων ενδιαφέροντος. Μπορεί να εκτελεστεί είτε χειροκίνητα είτε με τη χρήση αυτοματοποιημένων εργαλείων.

Δεδομένου ότι υπάρχει σχεδόν άπειρος αριθμός πιθανών δοκιμών ακόμη και για απλά στοιχεία λογισμικού, όλες οι δοκιμές λογισμικού χρησιμοποιούν κάποια στρατηγική για την επιλογή δοκιμών που είναι κατάλληλες για τον διαθέσιμο χρόνο και τους διαθέσιμους πόρους.

Η διαδικασία των δοκιμών είναι επαναληπτική, πράγμα που σημαίνει ότι η διόρθωση ενός σφάλματος μπορεί να αποκαλύψει άλλα που είναι βαθύτερα ή να δημιουργήσει νέα.

\subsection{Οφέλη των δοκιμών λογισμικού} 

Ακολουθούν ορισμένα από τα οφέλη των δοκιμών λογισμικού:
\begin{description}
	\item[Εξοικονόμηση πόρων:] Ο έγκαιρος προσδιορισμών των αστοχιών και ελλείψεων του λογισμικού οδηγεί στη γρήγορη διόρθωσή του, κάτι που εξοικονομεί χρόνο ανάπτυξης.
	\item[Ασφάλεια:] Οι τεχνικές δοκιμών ασφαλείας χρησιμοποιούνται για τον προσδιορισμό του επιπέδου ασφαλείας μιας εφαρμογής και οι δοκιμαστές προσπαθούν να εντοπίσουν τυχόν ευπάθειες που ενδέχεται να θέσουν σε κίνδυνο την ασφάλειά της. Ο στόχος είναι να εντοπιστούν και να μετριαστούν τυχόν ελαττώματα ή κενά ασφαλείας της εφαρμογής.
	\item[Διασφάλιση ποιότητας:] Η ποιότητα ενός προϊόντος μπορεί να διατηρηθεί μόνο όταν είναι απαλλαγμένο από σφάλματα και ανταποκρίνεται σε όλες τις απαιτήσεις των χρηστών.
	\item[Επιτάχυνση ανάπτυξης:] Οι δοκιμές λογισμικού επιταχύνουν τη διαδικασία ανάπτυξης με τον εντοπισμό ελαττωμάτων. Η ανίχνευση σε πρώιμο στάδιο διευκολύνει τη διόρθωση χωρίς να επηρεάζονται αρνητικά άλλες λειτουργίες του συστήματος.
\end{description}

\subsection{Τύποι δοκιμών λογισμικού}

Υπάρχουν διάφοροι τύποι δοκιμών λογισμικού, καθένας από τους οποίους έχει μοναδικούς στόχους και στρατηγικές.

\begin{description}
	\item[Αποδοχής:] Επαληθεύουν αν το σύστημα λειτουργεί όπως προβλέπεται.
	\item[Ολοκλήρωσης:] Εξετάζουν τη δυνατότητα των διάφορων μονάδων του συστήματος να λειτουργούν μαζί.
	\item[Μονάδας:] Επικυρώνουν ότι κάθε μονάδα λογισμικού λειτουργεί όπως αναμένεται. Μια μονάδα είναι το μικρότερο ελέγξιμο συστατικό μιας εφαρμογής.
	\item[Λειτουργική:] Ελέγχουν τις επιχειρηματικές λειτουργίες, με βάση τις λειτουργικές απαιτήσεις.
	\item[Επιδόσεων:] Διασφαλίζουν ότι το λογισμικό πληροί τις απαιτήσεις επιδόσεων, όπως ο χρόνος απόκρισης και η χρήση μνήμης.
	\item[Καταπόνησης:] Δοκιμάζουν τη μέγιστη καταπόνηση που μπορεί να αντέξει το σύστημα πριν αποτύχει.
	\item[Ευχρηστίας:] Επικυρώνουν την ικανότητα του πελάτη να χρησιμοποιεί αποτελεσματικά ένα σύστημα ή μια εφαρμογή.
\end{description}

Η επικύρωση των βασικών απαιτήσεων αποτελεί κρίσιμο κομμάτι της αξιολόγησης ενός προϊόντος λογισμικού. Εξίσου σημαντική, η διερευνητική δοκιμή βοηθά στην αποκάλυψη απρόβλεπτων σεναρίων που μπορεί να οδηγήσουν σε σφάλματα λογισμικού.

\section{Πλάνο Ελέγχου Ορθής Λειτουργίας}
Για το πλάνο ελέγχου ορθής λειτουργίας της πλατφόρμας, θα πρέπει να εξακριβωθεί πως ικανοποιούνται οι περιπτώσεις χρήσεις που έχουν οριστεί στην ενότητα \ref{use-cases}.

Ονομαστικά, οι έλεγχοι που πραγματοποιήθηκαν είναι οι παρακάτω:

\begin{itemize}
	\item \nameref{tab:test-cases-basic}
	\item \nameref{tab:test-cases-labs}
	\item \nameref{tab:test-cases-groups}
	\item \nameref{tab:test-cases-computers}
	\item \nameref{tab:test-cases-ipxeblocks}
	\item \nameref{tab:test-cases-boot-menus}
	\item \nameref{tab:test-cases-schedules}
	\item \nameref{tab:test-cases-logs}
\end{itemize}

Για την πραγματοποίηση των ελέγχων ορθής χρήσης κάθε λειτουργίας που περιγράφεται στις περιπτώσεις χρήσης της πλατφόρμας, έγινε μια εγκατάσταση της εφαρμογής σε δοκιμαστικό περιβάλλον, το οποίο όμως σχεδιάστηκε ώστε να πληρεί όλες τις προδιαγραφές παραγωγικής λειτουργίας, όπως για παράδειγμα ρύθμιση HTTPS, αποστολής email κλπ. Στη συνέχεια δημιουργήθηκαν οι κατάλληλοι χειριστές και περάστηκαν δοκιμαστικά στοιχεία στη βάση δεδομένων.

Στους πίνακες που ακολουθούν παρουσιάζονται οι έλεγχοι ορθής χρήσης που πραγματοποιήθηκαν για κάθε λειτουργία της πλατφόρμας, καθώς και τα αποτελέσματά τους.

\FloatBarrier
\subsection{Βασικές Λειτουργίες Χρηστών}

%
% Βασικές Λειτουργίες Χρηστών
%
\begin{longtable}{|p{0.44\linewidth}|p{0.35\linewidth}|p{0.11\linewidth}|}
	\caption{Έλεγχοι Βασικών Λειτουργιών Χρηστών} \label{tab:test-cases-basic} \\
	\hline \multicolumn{1}{|c|}{\textbf{Περιγραφή Ελέγχου}} & \multicolumn{1}{c|}{\textbf{Χειριστές}} & \multicolumn{1}{c|}{\textbf{Αποτέλεσμα}} \\ \hline \endfirsthead
	\caption[{}]{Έλεγχοι Βασικών Λειτουργιών Χρηστών (συνέχεια)} \\
	\hline \multicolumn{1}{|c|}{\textbf{Περιγραφή Ελέγχου}} & \multicolumn{1}{c|}{\textbf{Χειριστές}} & \multicolumn{1}{c|}{\textbf{Αποτέλεσμα}} \\ \hline \endhead \endfoot
	Εγγραφή χρήστη στο σύστημα & Ανώνυμος Χρήστης & \multicolumn{1}{c|}{\ding{51}} \\  \hline
	Σύνδεση χρήστη στην πλατφόρμα & Γενικός Διαχειριστής Πλατφόρμας, Διαχειριστής Εργαστηρίου & \multicolumn{1}{c|}{\ding{51}} \\  \hline
	Επανέκδοση κωδικού χρήστη & Γενικός Διαχειριστής Πλατφόρμας, Διαχειριστής Εργαστηρίου & \multicolumn{1}{c|}{\ding{51}} \\  \hline
	Υπενθύμιση Ονόματος Χρήστη & Γενικός Διαχειριστής Πλατφόρμας, Διαχειριστής Εργαστηρίου & \multicolumn{1}{c|}{\ding{51}} \\  \hline
	Αποσύνδεση από το σύστημα & Γενικός Διαχειριστής Πλατφόρμας, Διαχειριστής Εργαστηρίου & \multicolumn{1}{c|}{\ding{51}} \\  \hline
	Αλλαγή στοιχείων χρήστη & Γενικός Διαχειριστής Πλατφόρμας, Διαχειριστής Εργαστηρίου & \multicolumn{1}{c|}{\ding{51}} \\  \hline
	Διαγραφή χρήστη από το σύστημα & Γενικός Διαχειριστής Πλατφόρμας, Διαχειριστής Εργαστηρίου & \multicolumn{1}{c|}{\ding{51}} \\  \hline
\end{longtable}

\subsection{Διαχείριση Εργαστηρίων}

%
% Διαχείριση Εργαστηρίων
%
\begin{longtable}{|p{0.44\linewidth}|p{0.35\linewidth}|p{0.11\linewidth}|}
	\caption{Έλεγχοι Διαχείρισης Εργαστηρίων} \label{tab:test-cases-labs} \\
	\hline \multicolumn{1}{|c|}{\textbf{Περιγραφή Ελέγχου}} & \multicolumn{1}{c|}{\textbf{Χειριστές}} & \multicolumn{1}{c|}{\textbf{Αποτέλεσμα}} \\ \hline \endfirsthead
	\caption[{}]{Έλεγχοι Διαχείρισης Εργαστηρίων (συνέχεια)} \\
	\hline \multicolumn{1}{|c|}{\textbf{Περιγραφή Ελέγχου}} & \multicolumn{1}{c|}{\textbf{Χειριστές}} & \multicolumn{1}{c|}{\textbf{Αποτέλεσμα}} \\ \hline \endhead \endfoot
	Εισαγωγή εργαστηρίου στο σύστημα & Γενικός Διαχειριστής Πλατφόρμας, Διαχειριστής Εργαστηρίου & \multicolumn{1}{c|}{\ding{51}} \\  \hline
	Διαγραφή εργαστηρίου από το σύστημα & Γενικός Διαχειριστής Πλατφόρμας & \multicolumn{1}{c|}{\ding{51}} \\  \hline
	Επεξεργασία αναθέσεων Διαχειριστών Εργαστηρίων & Γενικός Διαχειριστής Πλατφόρμας & \multicolumn{1}{c|}{\ding{51}} \\  \hline
\end{longtable}

\subsection{Διαχείριση Ομάδων Υπολογιστών}

%
% Διαχείριση Ομάδων Υπολογιστών
%
\begin{longtable}{|p{0.44\linewidth}|p{0.35\linewidth}|p{0.11\linewidth}|}
	\caption{Έλεγχοι Διαχείρισης Ομάδων Υπολογιστών} \label{tab:test-cases-groups} \\
	\hline \multicolumn{1}{|c|}{\textbf{Περιγραφή Ελέγχου}} & \multicolumn{1}{c|}{\textbf{Χειριστές}} & \multicolumn{1}{c|}{\textbf{Αποτέλεσμα}} \\ \hline \endfirsthead
	\caption[{}]{Έλεγχοι Διαχείρισης Ομάδων Υπολογιστών (συνέχεια)} \\
	\hline \multicolumn{1}{|c|}{\textbf{Περιγραφή Ελέγχου}} & \multicolumn{1}{c|}{\textbf{Χειριστές}} & \multicolumn{1}{c|}{\textbf{Αποτέλεσμα}} \\ \hline \endhead \endfoot
	Δημιουργία ομάδας υπολογιστών & Γενικός Διαχειριστής Πλατφόρμας, Διαχειριστής Εργαστηρίου & \multicolumn{1}{c|}{\ding{51}} \\  \hline
	Διαγραφή ομάδας υπολογιστών & Γενικός Διαχειριστής Πλατφόρμας & \multicolumn{1}{c|}{\ding{51}} \\  \hline
	Επεξεργασία αναθέσεων Διαχειριστών Εργαστηρίων & Γενικός Διαχειριστής Πλατφόρμας & \multicolumn{1}{c|}{\ding{51}} \\  \hline
\end{longtable}

\subsection{Διαχείριση Υπολογιστών}

%
% Διαχείριση Υπολογιστών
%
\begin{longtable}{|p{0.44\linewidth}|p{0.35\linewidth}|p{0.11\linewidth}|}
	\caption{Έλεγχοι Διαχείρισης Υπολογιστών} \label{tab:test-cases-computers} \\
	\hline \multicolumn{1}{|c|}{\textbf{Περιγραφή Ελέγχου}} & \multicolumn{1}{c|}{\textbf{Χειριστές}} & \multicolumn{1}{c|}{\textbf{Αποτέλεσμα}} \\ \hline \endfirsthead
	\caption[{}]{Έλεγχοι Διαχείρισης Υπολογιστών (συνέχεια)} \\
	\hline \multicolumn{1}{|c|}{\textbf{Περιγραφή Ελέγχου}} & \multicolumn{1}{c|}{\textbf{Χειριστές}} & \multicolumn{1}{c|}{\textbf{Αποτέλεσμα}} \\ \hline \endhead \endfoot
	Εισαγωγή υπολογιστή στο σύστημα & Γενικός Διαχειριστής Πλατφόρμας, Διαχειριστής Εργαστηρίου & \multicolumn{1}{c|}{\ding{51}} \\  \hline
	Ανάθεση υπολογιστή σε εργαστήριο & Γενικός Διαχειριστής Πλατφόρμας, Διαχειριστής Εργαστηρίου & \multicolumn{1}{c|}{\ding{51}} \\  \hline
	Αφαίρεση υπολογιστή από εργαστήριο & Γενικός Διαχειριστής Πλατφόρμας, Διαχειριστής Εργαστηρίου & \multicolumn{1}{c|}{\ding{51}} \\  \hline
	Επεξεργασία στοιχείων υπολογιστή & Γενικός Διαχειριστής Πλατφόρμας, Διαχειριστής Εργαστηρίου & \multicolumn{1}{c|}{\ding{51}} \\  \hline
	Διαγραφή υπολογιστή από το σύστημα & Γενικός Διαχειριστής Πλατφόρμας, Διαχειριστής Εργαστηρίου & \multicolumn{1}{c|}{\ding{51}} \\  \hline
\end{longtable}

\subsection{Διαχείριση Ονοματισμένων Block Εντολών Τύπου iPXE}

%
% Διαχείριση Ονοματισμένων Block Εντολών Τύπου iPXE
%
\begin{longtable}{|p{0.44\linewidth}|p{0.35\linewidth}|p{0.11\linewidth}|}
	\caption{Έλεγχοι Διαχείρισης Ονοματισμένων Block Εντολών Τύπου iPXE} \label{tab:test-cases-ipxeblocks} \\
	\hline \multicolumn{1}{|c|}{\textbf{Περιγραφή Ελέγχου}} & \multicolumn{1}{c|}{\textbf{Χειριστές}} & \multicolumn{1}{c|}{\textbf{Αποτέλεσμα}} \\ \hline \endfirsthead
	\caption[{}]{Έλεγχοι Διαχείρισης Ονοματισμένων Block Εντολών Τύπου iPXE (συνέχεια)} \\
	\hline \multicolumn{1}{|c|}{\textbf{Περιγραφή Ελέγχου}} & \multicolumn{1}{c|}{\textbf{Χειριστές}} & \multicolumn{1}{c|}{\textbf{Αποτέλεσμα}} \\ \hline \endhead \endfoot
	Εισαγωγή block εντολών τύπου iPXE & Γενικός Διαχειριστής Πλατφόρμας, Διαχειριστής Εργαστηρίου & \multicolumn{1}{c|}{\ding{51}} \\  \hline
	Επεξεργασία στοιχείων block εντολών τύπου iPXE & Γενικός Διαχειριστής Πλατφόρμας, Διαχειριστής Εργαστηρίου & \multicolumn{1}{c|}{\ding{51}} \\  \hline
	Διαγραφή block εντολών τύπου iPXE & Γενικός Διαχειριστής Πλατφόρμας, Διαχειριστής Εργαστηρίου & \multicolumn{1}{c|}{\ding{51}} \\  \hline
\end{longtable}

\subsection{Διαχείριση Μενού Εκκίνησης}

%
% Διαχείριση Μενού Εκκίνησης
%
\begin{longtable}{|p{0.44\linewidth}|p{0.35\linewidth}|p{0.11\linewidth}|}
	\caption{Έλεγχοι Διαχείρισης Μενού Εκκίνησης} \label{tab:test-cases-boot-menus} \\
	\hline \multicolumn{1}{|c|}{\textbf{Περιγραφή Ελέγχου}} & \multicolumn{1}{c|}{\textbf{Χειριστές}} & \multicolumn{1}{c|}{\textbf{Αποτέλεσμα}} \\ \hline \endfirsthead
	\caption[{}]{Έλεγχοι Διαχείρισης Μενού Εκκίνησης (συνέχεια)} \\
	\hline \multicolumn{1}{|c|}{\textbf{Περιγραφή Ελέγχου}} & \multicolumn{1}{c|}{\textbf{Χειριστές}} & \multicolumn{1}{c|}{\textbf{Αποτέλεσμα}} \\ \hline \endhead \endfoot
	Δημιουργία μενού εκκίνησης & Γενικός Διαχειριστής Πλατφόρμας, Διαχειριστής Εργαστηρίου & \multicolumn{1}{c|}{\ding{51}} \\  \hline
	Επεξεργασία στοιχείων μενού εκκίνησης & Γενικός Διαχειριστής Πλατφόρμας, Διαχειριστής Εργαστηρίου & \multicolumn{1}{c|}{\ding{51}} \\  \hline
	Διαγραφή μενού εκκίνησης & Γενικός Διαχειριστής Πλατφόρμας, Διαχειριστής Εργαστηρίου & \multicolumn{1}{c|}{\ding{51}} \\  \hline
\end{longtable}


\subsection{Διαχείριση Χρονοδιαγραμμάτων}

%
% Διαχείριση Χρονοδιαγραμμάτων
%
\begin{longtable}{|p{0.44\linewidth}|p{0.35\linewidth}|p{0.11\linewidth}|}
	\caption{Έλεγχοι Χρονοδιαγραμμάτων} \label{tab:test-cases-schedules} \\
	\hline \multicolumn{1}{|c|}{\textbf{Περιγραφή Ελέγχου}} & \multicolumn{1}{c|}{\textbf{Χειριστές}} & \multicolumn{1}{c|}{\textbf{Αποτέλεσμα}} \\ \hline \endfirsthead
	\caption[{}]{Έλεγχοι Χρονοδιαγραμμάτων (συνέχεια)} \\
	\hline \multicolumn{1}{|c|}{\textbf{Περιγραφή Ελέγχου}} & \multicolumn{1}{c|}{\textbf{Χειριστές}} & \multicolumn{1}{c|}{\textbf{Αποτέλεσμα}} \\ \hline \endhead \endfoot
	Δημιουργία χρονοδιαγράμματος & Γενικός Διαχειριστής Πλατφόρμας, Διαχειριστής Εργαστηρίου & \multicolumn{1}{c|}{\ding{51}} \\  \hline
	Επεξεργασία στοιχείων χρονοδιαγράμματος & Γενικός Διαχειριστής Πλατφόρμας, Διαχειριστής Εργαστηρίου & \multicolumn{1}{c|}{\ding{51}} \\  \hline
	Διαγραφή χρονοδιαγράμματος & Γενικός Διαχειριστής Πλατφόρμας, Διαχειριστής Εργαστηρίου & \multicolumn{1}{c|}{\ding{51}} \\  \hline
\end{longtable}

\subsection{Διαχείριση Αρχείων Καταγραφής Συμβάντων}

%
% Διαχείριση Αρχείων Καταγραφής Συμβάντων
%
\begin{longtable}{|p{0.44\linewidth}|p{0.35\linewidth}|p{0.11\linewidth}|}
	\caption{Έλεγχοι Αρχείων Καταγραφής Συμβάντων} \label{tab:test-cases-logs} \\
	\hline \multicolumn{1}{|c|}{\textbf{Περιγραφή Ελέγχου}} & \multicolumn{1}{c|}{\textbf{Χειριστές}} & \multicolumn{1}{c|}{\textbf{Αποτέλεσμα}} \\ \hline \endfirsthead
	\caption[{}]{Έλεγχοι Αρχείων Καταγραφής Συμβάντων (συνέχεια)} \\
	\hline \multicolumn{1}{|c|}{\textbf{Περιγραφή Ελέγχου}} & \multicolumn{1}{c|}{\textbf{Χειριστές}} & \multicolumn{1}{c|}{\textbf{Αποτέλεσμα}} \\ \hline \endhead \endfoot
	Δημιουργία αρχείων καταγραφής συμβάντων & Γενικός Διαχειριστής Πλατφόρμας & \multicolumn{1}{c|}{\ding{51}} \\  \hline
	Διαγραφή αρχείων καταγραφής συμβάντων & Γενικός Διαχειριστής Πλατφόρμας & \multicolumn{1}{c|}{\ding{51}} \\  \hline
\end{longtable}

\FloatBarrier

\section{Μελέτη Κλιμάκωσης}
Τελικό στάδιο αξιολόγησης της πλατφόρμας που αναπτύχθηκε, ήταν η δοκιμή της σε πραγματικές συνθήκες και με μεγάλο όγκο δεδομένων.

Αποφασίστηκε η εφαρμογή να χρησιμοποιηθεί για την κάλυψη των αναγκών ενός workshop, στο οποίο οι 50 συμμετέχοντες θα έπρεπε να έχουν πανομοιότυπο περιβάλλον εργασίας με προεγκατεστημένα και προρυθμισμένα εργαλεία. Για το σκοπό αυτό, δημιουργήθηκε μια παραμετροποιημένη εικόνα του λειτουργικού συστήματος Linux Mint, στην οποία είχαν γίνει όλες οι απαραίτητες εγκαταστάσεις και ρυθμίσεις. Η εικόνα τοποθετήθηκε σε έναν TFTP server, στο τοπικό δίκτυο του εργαστηρίου όπου έλαβε χώρα το workshop. Ο DHCP server του εργαστηρίου ρυθμίστηκε επίσης ώστε να παραπέμπει τους υπολογιστές που εκκινούνταν, στην παραγωγική εγκατάσταση της εφαρμογής.

Νωρίτερα από το workshop είχε γίνει μια εκκίνηση του κάθε υπολογιστή, ώστε τα στοιχεία του (UUID και διεύθυνση MAC) να αποθηκευτούν στην πλατφόρμα και να είναι έτοιμοι για ανάθεση σε ομάδες. Δημιουργήθηκε μια ομάδα με την κατάλληλη διαδρομή και πρόθεμα για τον TFTP server του εργαστηρίου, στην οποία και έγινε ανάθεση όλων των υπολογιστών. Επίσης, δημιουργήθηκε μια εγγραφή μπλοκ εντολών τύπου iPXE, κατάλληλου για την τροποποιημένη εικόνα λειτουργικού συστήματος με την οποία έπρεπε να εκκινηθούν οι υπολογιστές, καθώς και ένα μενού εκκίνησης, το οποίο περιείχε το συγκεκριμένο μπλοκ εντολών. Τέλος, δημιουργήθηκε ένα χρονοδιάγραμμα για την ημερομηνία και ώρα του εργαστηρίου, στο οποίο καθοριζόταν πως στους υπολογιστές της ομάδας θα έπρεπε να σταλεί το συγκεκριμένο μενού εκκίνησης.

Τα αποτελέσματα ήταν πολύ θετικά, με τους υπολογιστές να εκκινούνται σωστά και χωρίς καθυστερήσεις. Αυτό απέδειξε έμπρακτα ότι η εφαρμογή είναι όχι μόνο λειτουργική, αλλά και αποδοτική σε μεγάλο όγκο δεδομένων.

\section{Σύνοψη Κεφαλαίου 5}
Στο κεφάλαιο 5, έγινε αναφορά στην αξιολόγηση του συστήματος μετά τις απαραίτητες δοκιμές. Σκοπός της αξιολόγησης του συστήματος είναι η εξαγωγή συμπερασμάτων σχετικά με την ορθή λειτουργία της πλατφόρμας και τις δυνατότητες επέκτασης και κλιμάκωσής της.

Τα συμπεράσματα, τα οφέλη και οι συστάσεις για μελλοντικές βελτιώσεις, όπως και η συνολική επισκόπηση της διπλωματικής εργασίας, παρατίθενται στο επόμενο κεφάλαιο.
