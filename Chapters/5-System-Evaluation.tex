\chapter{Αξιολόγηση Συστήματος}
Σε αυτό το κεφάλαιο, αναλύονται λεπτομερώς οι δοκιμές που έγιναν για την εξακρίβωση της ορθής λειτουργίας της δικτυακής εφαρμογής. Επίσης, καταγράφεται το πλάνο ελέγχου ορθής λειτουργίας της πλατφόρμας και γίνεται μελέτη κλιμάκωσης του συστήματος.

\section{Δοκιμή Πλατφόρμας}


\section{Πλάνο Ελέγχου Ορθής Λειτουργίας}
Για το πλάνο ελέγχου ορθής λειτουργίας της πλατφόρμας, θα πρέπει να εξακριβωθεί πως ικανοποιούνται οι περιπτώσεις χρήσεις που έχουν οριστεί στην ενότητα \ref{use-cases}.

Ονομαστικά, οι έλεγχοι που πραγματοποιήθηκαν είναι οι παρακάτω:

\begin{itemize}
	\item \nameref{tab:test-cases-basic}
	\item \nameref{tab:test-cases-labs}
	\item \nameref{tab:test-cases-groups}
	\item \nameref{tab:test-cases-computers}
	\item \nameref{tab:test-cases-ipxeblocks}
	\item \nameref{tab:test-cases-boot-menus}
	\item \nameref{tab:test-cases-schedules}
	\item \nameref{tab:test-cases-logs}
\end{itemize}

Στους πίνακες που ακολουθούν παρουσιάζονται οι έλεγχοι ορθής χρήσης που πραγματοποιήθηκαν για κάθε λειτουργία της πλατφόρμας, καθώς και τα αποτελέσματά τους.

\FloatBarrier
\subsection{Βασικές Λειτουργίες Χρηστών}

%
% Βασικές Λειτουργίες Χρηστών
%
\begin{longtable}{|p{0.44\linewidth}|p{0.35\linewidth}|p{0.11\linewidth}|}
	\caption{Έλεγχοι Βασικών Λειτουργιών Χρηστών} \label{tab:test-cases-basic} \\
	\hline \multicolumn{1}{|c|}{\textbf{Περιγραφή Ελέγχου}} & \multicolumn{1}{c|}{\textbf{Χειριστές}} & \multicolumn{1}{c|}{\textbf{Αποτέλεσμα}} \\ \hline \endfirsthead
	\caption[{}]{Έλεγχοι Βασικών Λειτουργιών Χρηστών (συνέχεια)} \\
	\hline \multicolumn{1}{|c|}{\textbf{Περιγραφή Ελέγχου}} & \multicolumn{1}{c|}{\textbf{Χειριστές}} & \multicolumn{1}{c|}{\textbf{Αποτέλεσμα}} \\ \hline \endhead \endfoot
	Εγγραφή χρήστη στο σύστημα & Ανώνυμος Χρήστης & \multicolumn{1}{c|}{\ding{51}} \\  \hline
	Σύνδεση χρήστη στην πλατφόρμα & Γενικός Διαχειριστής Πλατφόρμας, Διαχειριστής Εργαστηρίου & \multicolumn{1}{c|}{\ding{51}} \\  \hline
	Επανέκδοση κωδικού χρήστη & Γενικός Διαχειριστής Πλατφόρμας, Διαχειριστής Εργαστηρίου & \multicolumn{1}{c|}{\ding{51}} \\  \hline
	Υπενθύμιση Ονόματος Χρήστη & Γενικός Διαχειριστής Πλατφόρμας, Διαχειριστής Εργαστηρίου & \multicolumn{1}{c|}{\ding{51}} \\  \hline
	Αποσύνδεση από το σύστημα & Γενικός Διαχειριστής Πλατφόρμας, Διαχειριστής Εργαστηρίου & \multicolumn{1}{c|}{\ding{51}} \\  \hline
	Αλλαγή στοιχείων χρήστη & Γενικός Διαχειριστής Πλατφόρμας, Διαχειριστής Εργαστηρίου & \multicolumn{1}{c|}{\ding{51}} \\  \hline
	Διαγραφή χρήστη από το σύστημα & Γενικός Διαχειριστής Πλατφόρμας, Διαχειριστής Εργαστηρίου & \multicolumn{1}{c|}{\ding{51}} \\  \hline
\end{longtable}

\subsection{Διαχείριση Εργαστηρίων}

%
% Διαχείριση Εργαστηρίων
%
\begin{longtable}{|p{0.44\linewidth}|p{0.35\linewidth}|p{0.11\linewidth}|}
	\caption{Έλεγχοι Διαχείρισης Εργαστηρίων} \label{tab:test-cases-labs} \\
	\hline \multicolumn{1}{|c|}{\textbf{Περιγραφή Ελέγχου}} & \multicolumn{1}{c|}{\textbf{Χειριστές}} & \multicolumn{1}{c|}{\textbf{Αποτέλεσμα}} \\ \hline \endfirsthead
	\caption[{}]{Έλεγχοι Διαχείρισης Εργαστηρίων (συνέχεια)} \\
	\hline \multicolumn{1}{|c|}{\textbf{Περιγραφή Ελέγχου}} & \multicolumn{1}{c|}{\textbf{Χειριστές}} & \multicolumn{1}{c|}{\textbf{Αποτέλεσμα}} \\ \hline \endhead \endfoot
	Εισαγωγή εργαστηρίου στο σύστημα & Γενικός Διαχειριστής Πλατφόρμας, Διαχειριστής Εργαστηρίου & \multicolumn{1}{c|}{\ding{51}} \\  \hline
	Διαγραφή εργαστηρίου από το σύστημα & Γενικός Διαχειριστής Πλατφόρμας & \multicolumn{1}{c|}{\ding{51}} \\  \hline
	Επεξεργασία αναθέσεων Διαχειριστών Εργαστηρίων & Γενικός Διαχειριστής Πλατφόρμας & \multicolumn{1}{c|}{\ding{51}} \\  \hline
\end{longtable}

\subsection{Διαχείριση Ομάδων Υπολογιστών}

%
% Διαχείριση Ομάδων Υπολογιστών
%
\begin{longtable}{|p{0.44\linewidth}|p{0.35\linewidth}|p{0.11\linewidth}|}
	\caption{Έλεγχοι Διαχείρισης Ομάδων Υπολογιστών} \label{tab:test-cases-groups} \\
	\hline \multicolumn{1}{|c|}{\textbf{Περιγραφή Ελέγχου}} & \multicolumn{1}{c|}{\textbf{Χειριστές}} & \multicolumn{1}{c|}{\textbf{Αποτέλεσμα}} \\ \hline \endfirsthead
	\caption[{}]{Έλεγχοι Διαχείρισης Ομάδων Υπολογιστών (συνέχεια)} \\
	\hline \multicolumn{1}{|c|}{\textbf{Περιγραφή Ελέγχου}} & \multicolumn{1}{c|}{\textbf{Χειριστές}} & \multicolumn{1}{c|}{\textbf{Αποτέλεσμα}} \\ \hline \endhead \endfoot
	Δημιουργία ομάδας υπολογιστών & Γενικός Διαχειριστής Πλατφόρμας, Διαχειριστής Εργαστηρίου & \multicolumn{1}{c|}{\ding{51}} \\  \hline
	Διαγραφή ομάδας υπολογιστών & Γενικός Διαχειριστής Πλατφόρμας & \multicolumn{1}{c|}{\ding{51}} \\  \hline
	Επεξεργασία αναθέσεων Διαχειριστών Εργαστηρίων & Γενικός Διαχειριστής Πλατφόρμας & \multicolumn{1}{c|}{\ding{51}} \\  \hline
\end{longtable}

\subsection{Διαχείριση Υπολογιστών}

%
% Διαχείριση Υπολογιστών
%
\begin{longtable}{|p{0.44\linewidth}|p{0.35\linewidth}|p{0.11\linewidth}|}
	\caption{Έλεγχοι Διαχείρισης Υπολογιστών} \label{tab:test-cases-computers} \\
	\hline \multicolumn{1}{|c|}{\textbf{Περιγραφή Ελέγχου}} & \multicolumn{1}{c|}{\textbf{Χειριστές}} & \multicolumn{1}{c|}{\textbf{Αποτέλεσμα}} \\ \hline \endfirsthead
	\caption[{}]{Έλεγχοι Διαχείρισης Υπολογιστών (συνέχεια)} \\
	\hline \multicolumn{1}{|c|}{\textbf{Περιγραφή Ελέγχου}} & \multicolumn{1}{c|}{\textbf{Χειριστές}} & \multicolumn{1}{c|}{\textbf{Αποτέλεσμα}} \\ \hline \endhead \endfoot
	Εισαγωγή υπολογιστή στο σύστημα & Γενικός Διαχειριστής Πλατφόρμας, Διαχειριστής Εργαστηρίου & \multicolumn{1}{c|}{\ding{51}} \\  \hline
	Ανάθεση υπολογιστή σε εργαστήριο & Γενικός Διαχειριστής Πλατφόρμας, Διαχειριστής Εργαστηρίου & \multicolumn{1}{c|}{\ding{51}} \\  \hline
	Αφαίρεση υπολογιστή από εργαστήριο & Γενικός Διαχειριστής Πλατφόρμας, Διαχειριστής Εργαστηρίου & \multicolumn{1}{c|}{\ding{51}} \\  \hline
	Επεξεργασία στοιχείων υπολογιστή & Γενικός Διαχειριστής Πλατφόρμας, Διαχειριστής Εργαστηρίου & \multicolumn{1}{c|}{\ding{51}} \\  \hline
	Διαγραφή υπολογιστή από το σύστημα & Γενικός Διαχειριστής Πλατφόρμας, Διαχειριστής Εργαστηρίου & \multicolumn{1}{c|}{\ding{51}} \\  \hline
\end{longtable}

\subsection{Διαχείριση Ονοματισμένων Block Εντολών Τύπου iPXE}

%
% Διαχείριση Ονοματισμένων Block Εντολών Τύπου iPXE
%
\begin{longtable}{|p{0.44\linewidth}|p{0.35\linewidth}|p{0.11\linewidth}|}
	\caption{Έλεγχοι Διαχείρισης Ονοματισμένων Block Εντολών Τύπου iPXE} \label{tab:test-cases-ipxeblocks} \\
	\hline \multicolumn{1}{|c|}{\textbf{Περιγραφή Ελέγχου}} & \multicolumn{1}{c|}{\textbf{Χειριστές}} & \multicolumn{1}{c|}{\textbf{Αποτέλεσμα}} \\ \hline \endfirsthead
	\caption[{}]{Έλεγχοι Διαχείρισης Ονοματισμένων Block Εντολών Τύπου iPXE (συνέχεια)} \\
	\hline \multicolumn{1}{|c|}{\textbf{Περιγραφή Ελέγχου}} & \multicolumn{1}{c|}{\textbf{Χειριστές}} & \multicolumn{1}{c|}{\textbf{Αποτέλεσμα}} \\ \hline \endhead \endfoot
	Εισαγωγή block εντολών τύπου iPXE & Γενικός Διαχειριστής Πλατφόρμας, Διαχειριστής Εργαστηρίου & \multicolumn{1}{c|}{\ding{51}} \\  \hline
	Επεξεργασία στοιχείων block εντολών τύπου iPXE & Γενικός Διαχειριστής Πλατφόρμας, Διαχειριστής Εργαστηρίου & \multicolumn{1}{c|}{\ding{51}} \\  \hline
	Διαγραφή block εντολών τύπου iPXE & Γενικός Διαχειριστής Πλατφόρμας, Διαχειριστής Εργαστηρίου & \multicolumn{1}{c|}{\ding{51}} \\  \hline
\end{longtable}

\subsection{Διαχείριση Μενού Εκκίνησης}

%
% Διαχείριση Μενού Εκκίνησης
%
\begin{longtable}{|p{0.44\linewidth}|p{0.35\linewidth}|p{0.11\linewidth}|}
	\caption{Έλεγχοι Διαχείρισης Μενού Εκκίνησης} \label{tab:test-cases-boot-menus} \\
	\hline \multicolumn{1}{|c|}{\textbf{Περιγραφή Ελέγχου}} & \multicolumn{1}{c|}{\textbf{Χειριστές}} & \multicolumn{1}{c|}{\textbf{Αποτέλεσμα}} \\ \hline \endfirsthead
	\caption[{}]{Έλεγχοι Διαχείρισης Μενού Εκκίνησης (συνέχεια)} \\
	\hline \multicolumn{1}{|c|}{\textbf{Περιγραφή Ελέγχου}} & \multicolumn{1}{c|}{\textbf{Χειριστές}} & \multicolumn{1}{c|}{\textbf{Αποτέλεσμα}} \\ \hline \endhead \endfoot
	Δημιουργία μενού εκκίνησης & Γενικός Διαχειριστής Πλατφόρμας, Διαχειριστής Εργαστηρίου & \multicolumn{1}{c|}{\ding{51}} \\  \hline
	Επεξεργασία στοιχείων μενού εκκίνησης & Γενικός Διαχειριστής Πλατφόρμας, Διαχειριστής Εργαστηρίου & \multicolumn{1}{c|}{\ding{51}} \\  \hline
	Διαγραφή μενού εκκίνησης & Γενικός Διαχειριστής Πλατφόρμας, Διαχειριστής Εργαστηρίου & \multicolumn{1}{c|}{\ding{51}} \\  \hline
\end{longtable}


\subsection{Διαχείριση Χρονοδιαγραμμάτων}

%
% Διαχείριση Χρονοδιαγραμμάτων
%
\begin{longtable}{|p{0.44\linewidth}|p{0.35\linewidth}|p{0.11\linewidth}|}
	\caption{Έλεγχοι Χρονοδιαγραμμάτων} \label{tab:test-cases-schedules} \\
	\hline \multicolumn{1}{|c|}{\textbf{Περιγραφή Ελέγχου}} & \multicolumn{1}{c|}{\textbf{Χειριστές}} & \multicolumn{1}{c|}{\textbf{Αποτέλεσμα}} \\ \hline \endfirsthead
	\caption[{}]{Έλεγχοι Χρονοδιαγραμμάτων (συνέχεια)} \\
	\hline \multicolumn{1}{|c|}{\textbf{Περιγραφή Ελέγχου}} & \multicolumn{1}{c|}{\textbf{Χειριστές}} & \multicolumn{1}{c|}{\textbf{Αποτέλεσμα}} \\ \hline \endhead \endfoot
	Δημιουργία χρονοδιαγράμματος & Γενικός Διαχειριστής Πλατφόρμας, Διαχειριστής Εργαστηρίου & \multicolumn{1}{c|}{\ding{51}} \\  \hline
	Επεξεργασία στοιχείων χρονοδιαγράμματος & Γενικός Διαχειριστής Πλατφόρμας, Διαχειριστής Εργαστηρίου & \multicolumn{1}{c|}{\ding{51}} \\  \hline
	Διαγραφή χρονοδιαγράμματος & Γενικός Διαχειριστής Πλατφόρμας, Διαχειριστής Εργαστηρίου & \multicolumn{1}{c|}{\ding{51}} \\  \hline
\end{longtable}

\subsection{Διαχείριση Αρχείων Καταγραφής Συμβάντων}

%
% Διαχείριση Αρχείων Καταγραφής Συμβάντων
%
\begin{longtable}{|p{0.44\linewidth}|p{0.35\linewidth}|p{0.11\linewidth}|}
	\caption{Έλεγχοι Αρχείων Καταγραφής Συμβάντων} \label{tab:test-cases-logs} \\
	\hline \multicolumn{1}{|c|}{\textbf{Περιγραφή Ελέγχου}} & \multicolumn{1}{c|}{\textbf{Χειριστές}} & \multicolumn{1}{c|}{\textbf{Αποτέλεσμα}} \\ \hline \endfirsthead
	\caption[{}]{Έλεγχοι Αρχείων Καταγραφής Συμβάντων (συνέχεια)} \\
	\hline \multicolumn{1}{|c|}{\textbf{Περιγραφή Ελέγχου}} & \multicolumn{1}{c|}{\textbf{Χειριστές}} & \multicolumn{1}{c|}{\textbf{Αποτέλεσμα}} \\ \hline \endhead \endfoot
	Δημιουργία αρχείων καταγραφής συμβάντων & Γενικός Διαχειριστής Πλατφόρμας & \multicolumn{1}{c|}{\ding{51}} \\  \hline
	Διαγραφή αρχείων καταγραφής συμβάντων & Γενικός Διαχειριστής Πλατφόρμας & \multicolumn{1}{c|}{\ding{51}} \\  \hline
\end{longtable}

\FloatBarrier

\section{Μελέτη Κλιμάκωσης}


\section{Σύνοψη Κεφαλαίου 5}
Στο κεφάλαιο 5, έγινε αναφορά στην αξιολόγηση του συστήματος μετά τις απαραίτητες δοκιμές. Σκοπός της αξιολόγησης του συστήματος είναι η εξαγωγή συμπερασμάτων σχετικά με την ορθή λειτουργία της πλατφόρμας και τις δυνατότητες επέκτασης και κλιμάκωσής της.

Τα συμπεράσματα, τα οφέλη και οι συστάσεις για μελλοντικές βελτιώσεις, όπως και η συνολική επισκόπηση της διπλωματικής εργασίας, παρατίθενται στο επόμενο κεφάλαιο.
