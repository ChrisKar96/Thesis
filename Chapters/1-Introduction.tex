\chapter{Εισαγωγή}
\section{Ορισμός του Προβλήματος}

\section{Κίνητρα και Στόχοι Υλοποίησης}

\section{Περιπτώσεις Παρόμοιων Εργαλείων}
\subsection{FOG Project}
Το FOG είναι ένα δωρεάν πρόγραμμα απεικόνισης υπολογιστών ανοικτού κώδικα για Linux, Mac OS X και διάφορες εκδόσεις των Windows (XP, Vista, 7, 8/8.1 και 10). Ενσωματώνει μερικές τεχνολογίες ανοικτού κώδικα με μια διεπαφή ιστού βασισμένη στην PHP. Το FOG χρησιμοποιεί μόνο TFTP και PXE- δεν χρειάζεται καθόλου CD ή δίσκους εκκίνησης \cite{FOG_Project_2020}. 

Επιπλέον, με το FOG, πολλοί οδηγοί δικτύου περιλαμβάνονται ήδη στον πυρήνα του πελάτη (vanilla linux). Εφόσον τα δεδομένα είναι κάτω από 40GB, το FOG επιτρέπει επίσης την αντιγραφή μιας εικόνας από ένα μηχάνημα με διαμέρισμα 80GB σε ένα μηχάνημα με σκληρό δίσκο 40GB. Επειδή το FOG επιτρέπει την πολλαπλή διανομή, γίνεται δυνατή η δημιουργία εικόνα πολλών υπολογιστών χρησιμοποιώντας μία μόνο ροή. 

Το FOG  διαθέτει δυνατότητες  όπως ένα περιβάλλον εκκίνησης για PXE (DHCP, iPXE, TFTP, γρήγορη λήψη HTTP μεγάλων αρχείων εκκίνησης όπως ο πυρήνας και το initrd). Χρησιμοποιεί snapins για τη διεξαγωγή εργασιών και σεναρίων στους πελάτες και την εγκατάσταση λογισμικού, διαχειρίζεται εκτυπωτές. 
Ακόμη μία από τις λειτουργίες του, είναι η παρακολούθηση της δραστηριότητας του χρήστη του υπολογιστή και αυτόματη απενεργοποίηση του μηχανήματος όταν εμφανίζονται χρονικά όρια αδράνειας και η μετονομασία του κεντρικού υπολογιστή και ένταξη του στον τομέα.

Επίσης, έχει εξελιχθεί σε λύση διαχείρισης δικτύου και απεικόνισης/κλωνοποίησης, ξεπερνώντας την απλή λύση απεικόνισης, διαθέτει Anti-Virus και έχει την δυνατότητα διαγραφής δίσκων, επαναφοράς διαγραμμένων αρχείων και σάρωσης για bad blocks \cite{FOG_2020}.

\subsection{CCboot}
Το CCBoot είναι μια εμπορική λύση εκκίνησης χωρίς δίσκο, κλειστού κώδικα, που απευθύνεται σε καφετέριες Διαδικτύου, σχολεία και εταιρείες.
Με τη βοήθεια της τεχνολογίας εκκίνησης PXE, το Thin Client Software Solution του CCboot επιτρέπει την απομακρυσμένη εκκίνηση iscsi και την εκκίνηση χωρίς δίσκο των Windows 7 και XP από ένα διακομιστή thin client στο LAN.
Επίσης, το CCboot είναι πολύ χρήσιμο για πράγματα όπως η σύνδεση τομέων, η γρήγορη ανάκτηση, η ενοποιημένη διαχείριση καθώς δεν χρειάζεται σκληρούς δίσκους, κάνει εύκολα αναβαθμίσεις, λειτουργεί τέλεια με τον τομέα των Windows, ελαχιστοποιεί την κατανάλωση ενέργειας και είναι εξοπλισμένο με καλές επιδόσεις κρυφής μνήμης, φυσική μνήμη και υποστήριξη κρυφής μνήμης SSD \cite{CCboot_2022}.

\section{Σύνοψη Διπλωματικής Εργασίας}

\section{Σύνοψη Κεφαλαίου 1}
Στο ανωτέρω κεφάλαιο αναλύεται ο σχεδιασμός, η ανάπτυξη και η χρήση μιας πλατφόρμας ανοικτού κώδικα για τον έλεγχο της εκκίνησης υπολογιστών μέσω δικτύου, καθώς και οι λόγοι κατασκευής και οι λειτουργίες της. Επίσης, γίνεται αναφορά στα κίνητρα και τους στόχους υλοποίησης της εφαρμογής καθώς και σε περιπτώσεις παρόμοιων εργαλείων.

Στα επόμενα κεφάλαια, γίνεται αναφορά στη συνολική δομή της διαδικτυακής εφαρμογής, καθώς και στα εργαλεία και τις μεθόδους που χρησιμοποιήθηκαν καθ' όλη τη διάρκεια της ανάπτυξης.
