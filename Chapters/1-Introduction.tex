\chapter{Εισαγωγή}
\section{Ορισμός του Προβλήματος}
Το πρόβλημα που προσπαθεί να επιλύσει η παρούσα διπλωματική εργασία, με τον σχεδιασμό και την υλοποίηση της διαδικτυακής εφαρμογής iBoot, είναι η δημιουργία παραμετροποιημένων μενού εκκίνησης υπολογιστών, βάση χρονοδιαγράμματος, για το εργαστήριο Ρομποτικής, Ενσωματωμένων και Ολοκληρωμένων Συστημάτων του Πανεπιστημίου Δυτικής Μακεδονίας.

Στην υφιστάμενη κατάσταση, η δημιουργία των μενού εκκίνησης γίνεται με χρήση κάποιου php script, στο οποίο οι αλλαγές γίνονται χειροκίνητα. Μετά από χρόνια προσθήκης εγγραφών και παραμετροποιήσεων, το εν λόγω script έχει γίνει δυσνόητο και δύσχρηστο, κάνοντας έτσι χρονοβόρα τη διαδικασία επεξεργασίας του κάθε φορά που πρέπει να γίνουν αλλαγές.

Μετά από εξέταση διάφορων εμπορικών λύσεων, έγινε αντιληπτό πως δεν υπάρχει κάποια λύση που να καλύπτει πλήρως τις ανάγκες του εργαστηρίου. Οι σχετικές εφαρμογές στοχεύουν στην κάλυψη αναγκών ίντερνετ καφέ και περιβαλλόντων εργασίας, στα οποία δεν υπάρχει τόσο η ανάγκη για δυνατότητες όπως χρονοπρογραμματισμό εκκίνησης ή δυναμική δημιουργία παραμετροποιημένων μενού εκκίνησης.

\section{Κίνητρα και Στόχοι Υλοποίησης}
Ενώ υπάρχουν αρκετές διαδικτυακές εφαρμογές διαχείρισης δικτυακής εκκίνησης υπολογιστών, καμιά δε φαίνεται να μπορεί να ικανοποιήσει πλήρως τις ανάγκες του εργαστηρίου. Κάποιες εφαρμογές υποστηρίζουν εκκίνηση με ένα μόνο μενού, ενώ άλλες έχουν δυνατότητα διαχωρισμού των συσκευών σε ομάδες και εκκίνηση της κάθε συσκευής με διαφορετικό μενού. Ορισμένες έχουν αρκετές περισσότερες δυνατότητες και εργαλεία, τα οποία όμως δε χρειάζεται το εργαστήριο, ενώ καμιά δεν προσφέρει ένα βασικό χαρακτηριστικό για το οποίο υπάρχει ανάγκη· τη δυνατότητα καθορισμού διαφορετικού μενού εκκίνησης, βάση ομάδας συσκευών αλλά και χρονοδιαγράμματος.

Έτσι γεννήθηκε η ιδέα δημιουργίας της διαδικτυακής εφαρμογής ανοικτού κώδικα `iBoot', η οποία θα καλύψει τα κενά των διαθέσιμων εμπορικών λύσεων και θα δώσει λύση στο πρόβλημα της διαχείρισης της δικτυακής εκκίνησης των υπολογιστών του εργαστηρίου. Για να πετύχει τους στόχους της, η εφαρμογή iBoot θα πρέπει να έχει τα παρακάτω χαρακτηριστικά:
\begin{description}
	\item [Ανοικτού Κώδικα:] Μια εφαρμογή ανοικτού κώδικα είναι πιο εύκολο να επεκταθεί, να ελεγχθεί για κενά ασφαλείας και η λειτουργία της να γίνει κατανοητή από τους χρήστες, τους διαχειριστές και όσους τη συντηρούν.
	\item [Ανάπτυξη με PHP, MySQL \& Bootstrap:] Η ανάπτυξη θα πρέπει να γίνει με σύγχρονα και ασφαλή εργαλεία και τεχνολογίες, τα οποία να είναι σε θέση να συντηρήσουν και να επεκτείνουν οι φοιτητές του τμήματος.
	\item [Απλή διεπαφή χρήστη:] Η διεπαφή χρήστη θα πρέπει να είναι απλή και η εφαρμογή θα πρέπει να μην έχει περισσότερα χαρακτηριστικά από όσα χρειάζονται στο εργαστήριο. Όσο περισσότερα γίνονται τα χαρακτηριστικά και οι δυνατότητες μιας εφαρμογής, τόσο πιο δύσκολη γίνεται η χρήση της και η περιήγηση σε αυτή, ενώ ταυτόχρονα μεγαλώνει η πιθανότητα ύπαρξης ευπαθειών προς εκμετάλλευση από κακόβουλους χρήστες.
	\item [Προσαρμοστική Εμφάνιση:] Όπως συμβαίνει σε όλες τις σύγχρονες σελίδες και εφαρμογές, θα πρέπει η διεπαφή χρήστη να προσαρμόζεται στο μέγεθος της οθόνης του χρήστη, ώστε να προσφέρει την καλύτερη δυνατή εμπειρία χρήσης, τόσο σε υπολογιστές, όσο και κινητές συσκευές όπως τηλέφωνα και τάμπλετ.
	\item [Ομαδοποίηση Υπολογιστών:] Η εφαρμογή θα πρέπει να υποστηρίζει την ομαδοποίηση υπολογιστών, για πιο ευέλικτη διαχείριση της εκκίνησής τους.
	\item [Χρονοπρογραμματισμός Εκκίνησης:] Θα πρέπει να υποστηρίζεται η εκκίνηση με διαφορετικό μενού εκκίνησης τις ώρες των εργαστηρίων μαθημάτων του τμήματος.
	\item [Παρακολούθηση Εκκίνησης:] Ο διαχειριστής θα πρέπει να μπορεί να παρακολουθεί τους υπολογιστές σχετικά με το ποιοι και πότε εκκινήθηκαν.
	\item [Δυναμικά Μενού Εκκίνησης:] Τα μενού εκκίνησης πρέπει να είναι δυναμικά, δηλαδή να παράγονται τη στιγμή που ζητούνται από κάποιον υπολογιστή, με γνώμονα την τρέχουσα ημερομηνία/ώρα και την ομάδα στην οποία βρίσκεται ο υπολογιστής. Επίσης, θα πρέπει να μπορούν να υποστούν επεξεργασία ανά πάσα στιγμή από τους διαχειριστές της εφαρμογής, πραγματοποιώντας αλλαγές στο ίδιο το μενού ή στα μπλοκ iPXE που το αποτελούν.
	\item[Ασφάλεια:] Η εφαρμογή θα πρέπει να αναπτυχθεί με βάση σύγχρονα πρότυπα και πρακτικές ασφαλείας, ώστε να διασφαλίζει τόσο την ακεραιότητα των περιεχόμενων δεδομένων της, όσο και την ιδιωτικότητα των χρηστών της.
\end{description}

\section{Περιπτώσεις Παρόμοιων Εργαλείων}
\subsection{FOG Project}
Το FOG (Free and Open-source Ghost) είναι ένα δωρεάν και ανοικτού κώδικα πρόγραμμα απεικόνισης δίσκων υπολογιστών. Υποστηρίζει Linux, Mac OS και διάφορες εκδόσεις των Windows (XP, Vista, 7, 8/8.1 και 10). Ενσωματώνει τεχνολογίες ανοικτού κώδικα με μια διεπαφή ιστού βασισμένη στην PHP. Το FOG χρησιμοποιεί μόνο TFTP και PXE για την εκκίνηση των υπολογιστών, οπότε εκείνοι δε χρειάζονται καθόλου CD ή δίσκους εκκίνησης \cite{FOG_Project_2020}.

Επιπλέον, με το FOG, πολλοί οδηγοί δικτύου περιλαμβάνονται ήδη στον πυρήνα του πελάτη (vanilla linux). Το FOG επιτρέπει επίσης την αντιγραφή μιας εικόνας από ένα μηχάνημα με μεγαλύτερο διαμέρισμα δίσκου σε ένα μηχάνημα με μικρότερο σκληρό δίσκο, εφόσον ο χώρος του τελικού δίσκου επαρκεί για την αποθήκευση των δεδομένων του πηγαίου διαμερίσματος. Τέλος, επειδή το FOG επιτρέπει την πολλαπλή διανομή, γίνεται δυνατή η μετάδοση εικόνας σε πολλούς υπολογιστές ταυτόχρονα, χρησιμοποιώντας μία μόνο ροή.

Το FOG διαθέτει δυνατότητες όμοιες με ένα περιβάλλον εκκίνησης PXE (DHCP, iPXE, TFTP, γρήγορη λήψη HTTP μεγάλων αρχείων εκκίνησης όπως ο πυρήνας και το initrd). Χρησιμοποιεί snapins για τη διεξαγωγή εργασιών και σεναρίων στους πελάτες, την εγκατάσταση λογισμικού και τη διαχείριση εκτυπωτών. Μερικές ακόμη από τις λειτουργίες του είναι η παρακολούθηση της δραστηριότητας του χρήστη του υπολογιστή, η αυτόματη απενεργοποίηση του μηχανήματος έπειτα από προκαθορισμένα χρονικά όρια αδράνειας και η μετονομασία του κεντρικού υπολογιστή και ένταξή του σε τομέα. Επίσης, έχει εξελιχθεί σε λύση διαχείρισης δικτύου και απεικόνισης/κλωνοποίησης, ξεπερνώντας την απλή λύση απεικόνισης, διαθέτει Anti-Virus και έχει την δυνατότητα διαγραφής δίσκων, επαναφοράς διαγραμμένων αρχείων και σάρωσης δίσκων για bad blocks \cite{FOG_2020}.

\subsection{CCboot}
Το CCBoot είναι μια κλειστού κώδικα, εμπορική λύση εκκίνησης υπολογιστών χωρίς δίσκο. Aπευθύνεται σε internet cafe, σχολεία και εταιρείες που θέλουν να εκμεταλλευτούν τα πλεονεκτήματα της εικονοποίησης δίσκων και της δικτυακής εκκίνησης υπολογιστών. Με τη βοήθεια της τεχνολογίας εκκίνησης PXE, το Thin Client Software Solution του CCboot επιτρέπει την απομακρυσμένη εκκίνηση iSCSI και την εκκίνηση χωρίς δίσκο των Windows 7 και XP από ένα διακομιστή thin client στο τοπικό δίκτυο LAN.

Το CCboot έχει ορισμένα ιδιαίτερα χρήσιμα χαρακτηριστικά, όπως η σύνδεση τομέων, η γρήγορη ανάκτηση και η ενοποιημένη διαχείριση συσκευών. Εφόσον δε χρειάζεται σκληρούς δίσκους, διευκολύνει τις αναβαθμίσεις και την ομοιογένεια περιβάλλοντος σε όλες τις συσκευές του δικτύου. Τέλος, διαλειτουργεί τέλεια με τον τομέα των Windows, ελαχιστοποιεί την κατανάλωση ενέργειας και είναι εξοπλισμένο με καλές επιδόσεις κρυφής μνήμης, φυσική μνήμη και υποστήριξη κρυφής μνήμης SSD \cite{CCboot_2022}.

\subsection{EigenBoot}
Το Eigenboot είναι ένα σύστημα διαχείρισης εικόνων εκκίνησης που συγκεντρώνει και βελτιώνει τη διαχείριση των υπολογιστών γραφείου. Επιτρέπει στους υπολογιστές να εκκινούνται από ένα δίκτυο, αντί για έναν τοπικό δίσκο ή κάποια αφαιρούμενη συσκευή.

Το Eigenboot παρέχει επεκτασιμότητα και απόδοση, αξιοποιώντας τις υπολογιστικές δυνατότητες των τελικών συσκευών. Υποστηρίζει τη δυναμική κατανομή πόρων επιφάνειας εργασίας και την εξισορρόπηση φορτίου, η οποία αποτελεί ένα από τα πιο βασικά χαρακτηριστικά του. Επιτρέπει την ομαδοποίηση επιφάνειας εργασίας και μπορεί να προσφέρει διαφορετικά λειτουργικά συστήματα εκκίνησης ανά ομάδα. Μπορεί να σχεδιαστεί έτσι ώστε να επιτρέπει στον χρήστη να επιλέγει ένα λειτουργικό σύστημα κατά την εκκίνηση. Είναι συμβατό τόσο με λειτουργικά συστήματα Windows όσο και με Linux \cite{EigenBoot_2023}.

\section{Σύνοψη Διπλωματικής Εργασίας}
Η διαδικτυακή εφαρμογή διαχείρισης μενού δικτυακής εκκίνησης υπολογιστών `iBoot', δημιουργήθηκε για να καλύψει τις ανάγκες του εργαστηρίου Ρομποτικής, Ενσωματωμένων και Ολοκληρωμένων Συστημάτων του Πανεπιστημίου Δυτικής Μακεδονίας, εφόσον δεν υπήρχαν εμπορικές λύσεις με τα αναζητούμενα χαρακτηριστικά και τις απαραίτητες δυνατότητες.

Η εφαρμογή δημιουργήθηκε με PHP στο back-end, HTML, CSS, Javascript και Bootstrap στο front-end και αποθηκεύει τα δεδομένα της σε MySQL σχεσιακή βάση δεδομένων. Η χρήση της εφαρμογής γίνεται μέσω της διεπαφής χρήστη ή του REST API που έχουν αναπτυχθεί. Ο κώδικας της εφαρμογής είναι ανοικτός και δημοσιευμένος, και διαμοιράζεται με άδεια χρήσης MIT.

Η εφαρμογή υποστηρίζει τρία (3) είδη χρηστών.
\begin{description}
	\item[Ανώνυμος Χρήστης:] Ο Ανώνυμος Χρήστης έχει πρόσβαση μόνο στις σελίδες σύνδεσης, εγγραφής και υπενθύμισης στοιχείων πρόσβασης.
	\item[Διαχειριστής Εργαστηρίου:] Ο Διαχειριστής Εργαστηρίου μπορεί να διαχειρίζεται υπολογιστές που βρίσκονται σε εργαστήρια που του έχουν ανατεθεί. Μπορεί επίσης να τοποθετεί σε εργαστήρια τα οποία διαχειρίζεται, υπολογιστές που δεν έχουν ήδη τοποθετηθεί σε άλλα εργαστήρια.
	\item[Διαχειριστής:] Ο Διαχειριστής μπορεί να χρησιμοποιήσει όλες τις δυνατότητες της πλατφόρμας χωρίς περιορισμούς. Μπορεί να προσθέτει, να αφαιρεί και να επεξεργάζεται υπολογιστές, ομάδες, εργαστήρια, μπλοκ iPXE, μενού εκκίνησης, χρονοδιαγράμματα και χρήστες. Μπορεί επίσης να προβάλει τα αρχεία καταγραφής συμβάντων της εφαρμογής.
\end{description}
Επίσης, οι υπολογιστές αλληλεπιδρούν με συγκεκριμένες σελίδες της εφαρμογής, για να εγγραφούν σε αυτή και για να λάβουν μενού εκκίνησης.

Έχουν ακολουθηθεί οι βέλτιστες πρακτικές και τα πιο σύγχρονα πρότυπα ασφαλείας στον σχεδιασμό και την υλοποίηση της εφαρμογής, με σκοπό την διασφάλιση της περιεχόμενης πληροφορίας. Οι κωδικοί πρόσβασης των χρηστών αποθηκεύονται στη βάση δεδομένων της εφαρμογής μόνο αφού περαστούν από κρυπτογραφικά ασφαλείς συναρτήσεις κατακερματισμού, και δεν εμφανίζονται ποτέ σε κανένα σημείο της εφαρμογής. Πριν τη φόρτωση κάθε σελίδας γίνεται έλεγχος αυθεντικοποίησης και δικαιωμάτων του χρήστη που θέλει να την προβάλει, ενώ και για το API ισχύει ακριβώς το ίδιο, με την επισύναψη των JWT tokens σε κάθε αίτημα. Τέλος, η εφαρμογή δύναται να λειτουργήσει τόσο με πρωτόκολλο HTTP, όσο και με το ασφαλέστερο HTTPS, δίνοντας μάλιστα δυνατότητα κρυπτογράφησης και των cookies της συνεδρίας.

Η εγκατάσταση της διαδικτυακής εφαρμογής `iBoot' στο εργαστήριο Ρομποτικής, Ενσωματωμένων και Ολοκληρωμένων Συστημάτων του Πανεπιστημίου Δυτικής Μακεδονίας, θα εκσυγχρονίσει τη διαδικασία διαχείρισης των μενού δικτυακής εκκίνησης των υπολογιστών του, και θα διευκολύνει σημαντικά την καταγραφή και την παρακολούθησή τους.

\section{Σύνοψη Κεφαλαίου 1}
Στο ανωτέρω κεφάλαιο αναλύεται ο σχεδιασμός, η ανάπτυξη και η χρήση μιας πλατφόρμας ανοικτού κώδικα για τον έλεγχο της εκκίνησης υπολογιστών μέσω δικτύου, καθώς και οι λόγοι κατασκευής και οι λειτουργίες της. Επίσης, γίνεται αναφορά στα κίνητρα και τους στόχους υλοποίησης της εφαρμογής καθώς και σε περιπτώσεις παρόμοιων εργαλείων.

Στα επόμενα κεφάλαια, γίνεται αναφορά στη συνολική δομή της διαδικτυακής εφαρμογής, καθώς και στα εργαλεία και τις μεθόδους που χρησιμοποιήθηκαν καθ' όλη τη διάρκεια της ανάπτυξης.
