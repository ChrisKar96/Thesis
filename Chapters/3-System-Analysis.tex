\chapter{Δομή \& Οργάνωση Συστήματος}

\section{Ανάλυση Συστήματος}
Για την επίτευξη της ορθής ανάλυσης και σχεδίασης του συστήματος, είναι απαραίτητη η κατανόηση των λειτουργιών του και των δυνατοτήτων που προσφέρει στον χρήστη-διαχειριστή του. Ο διαχειριστής του συστήματος έχει τη δυνατότητα να προσθέτει, να αφαιρεί και να επεξεργάζεται στοιχεία υπολογιστών, οι οποίοι δύναται να εκκινηθούν δικτυακά, με μενού επιλογής λειτουργικού συστήματος παρεχόμενο από το σύστημα. Ο διαχειριστής μπορεί να προσθέσει και να διαχειριστεί κτήρια και δωμάτια στα οποία βρίσκονται οι υπολογιστές. Μια ακόμη ενέργεια στην οποία μπορεί να προβεί ο διαχειριστής, είναι να τοποθετεί τους υπολογιστές σε ομάδες, οι οποίες καθορίζουν το μενού λειτουργικών συστημάτων που θα παρέχει το σύστημα, ανάλογα με την τρέχουσα ημερομηνία / ώρα. Τέλος, ο διαχειριστής ορίζει τα διαθέσιμα λειτουργικά συστήματα, φτιάχνει μενού με χρήση των ορισμένων λειτουργικών συστημάτων και ρυθμίζει τις ημερομηνίες / ώρες στις οποίες το σύστημα θα παρέχει τα μενού λειτουργικών συστημάτων στις ομάδες υπολογιστών.
 
\section{Ανάλυση Απαιτήσεων}

\subsection{Λειτουργικές Προδιαγραφές}
Η διαδικτυακή εφαρμογή που υλοποιήθηκε, δίνει τη δυνατότητα στον χρήστη να διαχειρίζεται τα μενού λειτουργικών συστημάτων κατά τη δικτυακή εκκίνηση ομάδων υπολογιστών.

\subsection{Μη-Λειτουργικές Προδιαγραφές}
Η διαδικτυακή εφαρμογή έχει υλοποιηθεί ως ένα web interface που επικοινωνεί με τη βάση δεδομένων μέσω ενός CRUD (Create - Read - Update - Delete) API, με χρήση του CodeIgniter PHP Framework.

\section{Περιπτώσεις Χρήσης}
Στους πίνακες που ακολουθούν παρουσιάζονται οι περιπτώσεις χρήσης της πλατφόρμας.

\begin{table}[h]
	\caption{Σύνδεση στο σύστημα}
	\label{tab:use-case-login}
	\begin{tabular}{|p{0.2\linewidth}|p{0.7\linewidth}|}
		\hline
		Περιγραφή: & Ο τρόπος με τον οποίο συνδέεται ο χρήστης στην πλατφόρμα \\ \hline
		Χειριστές: & Γενικός Διαχειριστής Πλατφόρμας, Διαχειριστής Εργαστηρίου \\ \hline
		Προϋποθέσεις: & Ο χρήστης πρέπει να είναι εγγεγραμμένος στην πλατφόρμα \\ \hline
		Βασική ροή: & 
		\begin{enumerate}
			\item Ο χρήστης δεν έχει session cookie στον περιηγητή ιστού του
			\item Ο χρήστης εισάγει το όνομα χρήστη και το κωδικό του λογαριασμού του
			\item Επιβεβαιώνεται η επιτυχής εισαγωγή των στοιχείων και ο χρήστης συνδέεται στο σύστημα
		\end{enumerate} \\ \hline
		Εναλλακτικό σενάριο: & O χρήστης εισάγει εσφαλμένα στοιχεία, η πλατφόρμα απορρίπτει τη σύνδεση, εμφανίζει μήνυμα σφάλματος και ζητά από τον χρήστη να δοκιμάσει ξανά να εισαγάγει τα σωστά στοιχεία. \\ \hline
	\end{tabular}
\end{table}

\begin{table}[h]
	\caption{Εγγραφή χρήστη στο σύστημα}
	\label{tab:use-case-register}
	\begin{tabular}{|p{0.2\linewidth}|p{0.7\linewidth}|}
		\hline
		Περιγραφή: & Ο τρόπος με τον οποίο εγγράφονται οι χρήστες στην πλατφόρμα \\ \hline
		Χειριστές: & Ανώνυμος Χρήστης \\ \hline
		Προϋποθέσεις: & Καμία \\ \hline
		Βασική ροή: & 
		\begin{enumerate}
			\item Ο χρήστης συμπληρώνει τα στοιχεία του στη σελίδα εγγραφής
			\item Επιβεβαιώνεται η επιτυχής εισαγωγή των στοιχείων και γίνεται επαλήθευσή τους από client-side και server-side validators για την εξακρίβωση τύπου και πλήθους δεδομένων
			\item Επιβεβαιώνεται η επιτυχής εισαγωγή των στοιχείων και εισέρχεται στο σύστημα
		\end{enumerate} \\ \hline
		Εναλλακτικό σενάριο: & O χρήστης εισάγει εσφαλμένα στοιχεία, η πλατφόρμα απορρίπτει τη σύνδεση, εμφανίζει μήνυμα σφάλματος και ζητά από τον χρήστη να δοκιμάσει ξανά να εισαγάγει τα σωστά στοιχεία.                     \\ \hline
	\end{tabular}
\end{table}

\begin{table}[h]
	\caption{Επανέκδοση κωδικού}
	\label{tab:use-case-forgot-password}
	\begin{tabular}{|p{0.2\linewidth}|p{0.7\linewidth}|}
		\hline
		Περιγραφή: & Ο τρόπος με τον οποίο ο χρήστης δημιουργεί νέο κωδικό για τον λογαριασμό του σε περίπτωση που ξεχάσει τον παλιό \\ \hline
		Χειριστές: & Γενικός Διαχειριστής Πλατφόρμας, Διαχειριστής Εργαστηρίου \\ \hline
		Προϋποθέσεις: & Καμία \\ \hline
		Βασική ροή: & 
		\begin{enumerate}
			\item Ο χρήστης έχει ξεχάσει τον κωδικό του λογαριασμού του
			\item Ο χρήστης περιηγείται στο σημείο επανέκδοσης κωδικού πρόσβασης
			\item Εισάγει το όνομα χρήστη του στη φόρμα
			\item Λαμβάνει στο email του έναν ειδικό σύνδεσμο επανέκδοσης του κωδικού του
			\item Ακολουθεί τον σύνδεσμο και εισάγει τον νέο κωδικού του λογαριασμού του
			\item Συνδέεται στην πλατφόρμα με χρήση του νέου κωδικού του
		\end{enumerate} \\ \hline
		Εναλλακτικό σενάριο: & Αν ο χρήστης εισάγει λάθος όνομα χρήστη στο βήμα 3, δε θα λάβει το email επαναφοράς του κωδικού του. \\ \hline
	\end{tabular}
\end{table}

\begin{table}[h]
	\caption{Υπενθύμιση Ονόματος Χρήστη}
	\label{tab:use-case-forgot-username}
	\begin{tabular}{|p{0.2\linewidth}|p{0.7\linewidth}|}
		\hline
		Περιγραφή: & Ο τρόπος με τον οποίο ο χρήστης ζητά email υπενθύμισης για το όνομα χρήστη του λογαριασμού του \\ \hline
		Χειριστές: & Γενικός Διαχειριστής Πλατφόρμας, Διαχειριστής Εργαστηρίου \\ \hline
		Προϋποθέσεις: & Καμία \\ \hline
		Βασική ροή: & 
		\begin{enumerate}
			\item Ο χρήστης έχει ξεχάσει το όνομα χρήστη του λογαριασμού του
			\item Ο χρήστης περιηγείται στο σημείο υπενθύμισης ονόματος χρήστη
			\item Εισάγει το email του λογαριασμού του
			\item Λαμβάνει στο email του υπενθύμιση του ονόματος χρήστη του
			\item Ακολουθεί τον σύνδεσμο και εισάγει τον νέο κωδικού του λογαριασμού του
			\item Συνδέεται στην πλατφόρμα με χρήση των στοιχείων σύνδεσης του
		\end{enumerate} \\ \hline
		Εναλλακτικό σενάριο: & Αν ο χρήστης εισάγει λάθος email στο βήμα 3, δε θα λάβει το email υπενθύμισης του ονόματος χρήστη του. \\ \hline
	\end{tabular}
\end{table}

\begin{table}[h]
	\caption{Αποσύνδεση}
	\label{tab:use-case-logout}
	\begin{tabular}{|p{0.2\linewidth}|p{0.7\linewidth}|}
		\hline
		Περιγραφή: & Ο τρόπος με τον οποίο ο χρήστης αποσυνδέεται από την πλατφόρμα \\ \hline
		Χειριστές: & Γενικός Διαχειριστής Πλατφόρμας, Διαχειριστής Εργαστηρίου \\ \hline
		Προϋποθέσεις: & Καμία \\ \hline
		Βασική ροή: & 
		\begin{enumerate}
			\item Επιλογή της Αποσύνδεσης από το κεντρικό μενού.
		\end{enumerate} \\ \hline
		Εναλλακτικό σενάριο: & \\ \hline
	\end{tabular}
\end{table}

\section{Σχεδιασμός Βάσης Δεδομένων}

\input{Chapters/3-4-Database-Schema}

\section{Επεξήγηση Αρχείων}

\section{Ασφάλεια Συστήματος}
