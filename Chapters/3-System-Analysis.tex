\chapter{Δομή \& Οργάνωση Συστήματος}

\section{Ανάλυση Συστήματος}
Για την επίτευξη της ορθής ανάλυσης και σχεδίασης του συστήματος, είναι απαραίτητη η κατανόηση των λειτουργιών του και των δυνατοτήτων που προσφέρει στον χρήστη-διαχειριστή του. Ο διαχειριστής του συστήματος έχει τη δυνατότητα να προσθέτει, να αφαιρεί και να επεξεργάζεται στοιχεία υπολογιστών, οι οποίοι δύναται να εκκινηθούν δικτυακά, με μενού επιλογής λειτουργικού συστήματος παρεχόμενο από το σύστημα. Ο διαχειριστής μπορεί να προσθέσει και να διαχειριστεί κτήρια και δωμάτια στα οποία βρίσκονται οι υπολογιστές. Μια ακόμη ενέργεια στην οποία μπορεί να προβεί ο διαχειριστής, είναι να τοποθετεί τους υπολογιστές σε ομάδες, οι οποίες καθορίζουν το μενού λειτουργικών συστημάτων που θα παρέχει το σύστημα, ανάλογα με την τρέχουσα ημερομηνία / ώρα. Τέλος, ο διαχειριστής ορίζει τα διαθέσιμα λειτουργικά συστήματα, φτιάχνει μενού με χρήση των ορισμένων λειτουργικών συστημάτων και ρυθμίζει τις ημερομηνίες / ώρες στις οποίες το σύστημα θα παρέχει τα μενού λειτουργικών συστημάτων στις ομάδες υπολογιστών.
 
\section{Ανάλυση Απαιτήσεων}

\subsection{Λειτουργικές προδιαγραφές}
Η διαδικτυακή εφαρμογή που υλοποιήθηκε, δίνει τη δυνατότητα στον χρήστη να διαχειρίζεται τα μενού λειτουργικών συστημάτων κατά τη δικτυακή εκκίνηση ομάδων υπολογιστών.

\subsection{Μη-Λειτουργικές προδιαγραφές}
Η διαδικτυακή εφαρμογή έχει υλοποιηθεί ως ένα web interface που επικοινωνεί με τη βάση δεδομένων μέσω ενός CRUD (Create - Read - Update - Delete) API, με χρήση του CodeIgniter PHP Framework.

\section{Περιπτώσεις Χρήσης}

\section{Σχεδιασμός βάσης δεδομένων}

\input{Chapters/3-4-Database-Schema}

\section{Επεξήγηση Αρχείων}

\section{Ασφάλεια Συστήματος}
