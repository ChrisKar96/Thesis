\chapter{Δομή \& Οργάνωση Συστήματος}
Σε αυτό το κεφάλαιο, περιγράφονται λεπτομερώς οι λειτουργικές και μη-λειτουργικές απαιτήσεις του συστήματος διαχείρισης εκκίνησης υπολογιστών μέσω δικτύου. Επίσης, αναλύονται τα δύο είδη χρηστών που έχουν πρόσβαση στο σύστημα και οι δυνατότητες που παρέχονται από τη διαδικτυακή εφαρμογή του συστήματος. Ακόμα, γίνεται ανάλυση του σχήματος της βάσης δεδομένων και επεξήγηση αρχείων του κώδικα, ο οποίος συντέλεσε στη δημιουργία του πληροφοριακού συστήματος.

\section{Ανάλυση Συστήματος}
Για την επίτευξη της ορθής ανάλυσης και σχεδίασης του συστήματος, είναι απαραίτητη η κατανόηση των λειτουργιών του και των δυνατοτήτων που προσφέρει στους χρήστες του. Οι χρήστες είναι είτε γενικοί διαχειριστές του συστήματος, είτε διαχειριστές συγκεκριμένων εργαστηρίων.

Οι γενικοί διαχειριστές του συστήματος, έχουν τη δυνατότητα να προσθέτουν, να αφαιρούν και να επεξεργάζονται στοιχεία υπολογιστών, οι οποίοι δύναται να εκκινηθούν δικτυακά, με μενού επιλογής λειτουργικού συστήματος παρεχόμενο από το σύστημα. Μπορούν ακόμα να προσθέτουν και να διαχειρίζονται εργαστήρια στα οποία βρίσκονται οι υπολογιστές. Μια ακόμη ενέργεια στην οποία μπορούν να προβούν οι γενικοί διαχειριστές, είναι να τοποθετούν τους υπολογιστές σε ομάδες, οι οποίες καθορίζουν τα μενού λειτουργικών συστημάτων που θα παρέχονται από το σύστημα, ανάλογα με την τρέχουσα ημερομηνία / ώρα. Τέλος, οι γενικοί διαχειριστές ορίζουν τα διαθέσιμα λειτουργικά συστήματα, φτιάχνουν μενού εκκίνησης με χρήση των ορισμένων λειτουργικών συστημάτων και ρυθμίζουν τις ημερομηνίες / ώρες στις οποίες το σύστημα θα παρέχει τα μενού λειτουργικών συστημάτων στις ομάδες υπολογιστών.

Οι διαχειριστές εργαστηρίων, μπορούν να πραγματοποιούν τι ίδιες ενέργειες με τους γενικούς διαχειριστές, αλλά μόνο για τα εργαστήρια που τους έχουν ανατεθεί. Μπορούν δηλαδή να διαχειρίζονται τους υπολογιστές, τις ομάδες υπολογιστών και τα μενού εκκίνησης των ομάδων, που βρίσκονται στα εργαστήρια για τα οποία έχουν λάβει δικαίωμα διαχείρισης από τους γενικούς διαχειριστές.
 
\section{Ανάλυση Απαιτήσεων}

\subsection{Λειτουργικές Προδιαγραφές}
Η διαδικτυακή εφαρμογή που υλοποιήθηκε, δίνει τη δυνατότητα στον χρήστη να διαχειρίζεται τα μενού λειτουργικών συστημάτων κατά τη δικτυακή εκκίνηση ομάδων υπολογιστών.

\subsection{Μη-Λειτουργικές Προδιαγραφές}
Η διαδικτυακή εφαρμογή έχει υλοποιηθεί ως ένα web interface που επικοινωνεί με τη βάση δεδομένων μέσω ενός CRUD (Create - Read - Update - Delete) API, με χρήση του CodeIgniter PHP Framework.

\section{Περιπτώσεις Χρήσης}
Στους πίνακες που ακολουθούν παρουσιάζονται οι περιπτώσεις χρήσης της πλατφόρμας.

\FloatBarrier
%
% Σύνδεση στο σύστημα
%
\begin{longtable}{|p{0.2\linewidth}|p{0.7\linewidth}|} 
	\caption{Σύνδεση στο σύστημα} \label{tab:use-case-login} \\
	\hline
	\endfirsthead
	\caption{Σύνδεση στο σύστημα (συνέχεια)} \\ 
	\endhead \endfoot 
	Περιγραφή: & Ο τρόπος με τον οποίο συνδέεται ο χρήστης στην πλατφόρμα \\ \hline
	Χειριστές: & Γενικός Διαχειριστής Πλατφόρμας, Διαχειριστής Εργαστηρίου \\ \hline
	Προϋποθέσεις: & Ο χρήστης πρέπει να είναι εγγεγραμμένος στην πλατφόρμα \\ \hline
	Βασική ροή: & 
	\begin{enumerate}
		\vspace{-1cm}
		\addtolength{\itemindent}{-0.4cm}
		\item Ο χρήστης δεν έχει cookie αυθεντικοποίησης χρήστη της εφαρμογής στον περιηγητή ιστού του
		\item Ο χρήστης εισάγει το όνομα χρήστη και το κωδικό του λογαριασμού του
		\item Επιβεβαιώνεται η επιτυχής εισαγωγή των στοιχείων
		\item O χρήστης συνδέεται στο σύστημα και αποθηκεύεται cookie αυθεντικοποίησης χρήστη της εφαρμογής στον περιηγητή ιστού του
		\vspace{-0.7cm}
	\end{enumerate} \\ \hline
	Εναλλακτικό σενάριο: & O χρήστης εισάγει εσφαλμένα στοιχεία, η πλατφόρμα απορρίπτει τη σύνδεση, εμφανίζει μήνυμα σφάλματος και ζητά από τον χρήστη να δοκιμάσει ξανά να εισαγάγει τα σωστά στοιχεία. \\ \hline
\end{longtable}

%
% Εγγραφή χρήστη στο σύστημα
%
\begin{longtable}{|p{0.2\linewidth}|p{0.7\linewidth}|} 
	\caption{Εγγραφή χρήστη στο σύστημα} \label{tab:use-case-signup} \\
	\hline
	\endfirsthead
	\caption{Εγγραφή χρήστη στο σύστημα (συνέχεια)} \\ 
	\endhead \endfoot 
	Περιγραφή: & Ο τρόπος με τον οποίο εγγράφονται οι χρήστες στην πλατφόρμα \\ \hline
	Χειριστές: & Ανώνυμος Χρήστης \\ \hline
	Προϋποθέσεις: & Καμία \\ \hline
	Βασική ροή: & 
	\begin{enumerate}
		\vspace{-1cm}
		\addtolength{\itemindent}{-0.4cm}
		\item Ο χρήστης συμπληρώνει τα στοιχεία του στη σελίδα εγγραφής
		\item Επιβεβαιώνεται η επιτυχής εισαγωγή των στοιχείων και γίνεται επαλήθευσή τους από client-side και server-side validators για την εξακρίβωση τύπου και πλήθους δεδομένων
		\item Η εγγραφή του χρήστη στο σύστημα ολοκληρώνεται, ο χρήστης μεταφέρεται στη σελίδα σύνδεσης και παράλληλα λαμβάνει email με σύνδεσμο επιβεβαίωσης της διεύθυνσης email του
		\vspace{-0.7cm}
	\end{enumerate} \\ \hline
	Εναλλακτικό σενάριο: & O χρήστης εισάγει εσφαλμένα στοιχεία, η πλατφόρμα απορρίπτει τη σύνδεση, εμφανίζει μήνυμα σφάλματος και ζητά από τον χρήστη να δοκιμάσει ξανά να εισαγάγει τα σωστά στοιχεία. \\ \hline
\end{longtable}

%
% Επανέκδοση κωδικού
%
\begin{longtable}{|p{0.2\linewidth}|p{0.7\linewidth}|} 
	\caption{Επανέκδοση κωδικού} \label{tab:use-case-forgot-password} \\
	\hline
	\endfirsthead
	\caption{Επανέκδοση κωδικού (συνέχεια)} \\ 
	\endhead \endfoot 
	Περιγραφή: & Ο τρόπος με τον οποίο ο χρήστης δημιουργεί νέο κωδικό για τον λογαριασμό του σε περίπτωση που ξεχάσει τον παλιό \\ \hline
	Χειριστές: & Γενικός Διαχειριστής Πλατφόρμας, Διαχειριστής Εργαστηρίου \\ \hline
	Προϋποθέσεις: & Καμία \\ \hline
	Βασική ροή: & 
	\begin{enumerate}
		\vspace{-1cm}
		\addtolength{\itemindent}{-0.4cm}
		\item Ο χρήστης έχει ξεχάσει τον κωδικό του λογαριασμού του
		\item Ο χρήστης περιηγείται στο σημείο επανέκδοσης κωδικού πρόσβασης
		\item Εισάγει το όνομα χρήστη του στη φόρμα
		\item Λαμβάνει στο email του έναν ειδικό σύνδεσμο επανέκδοσης του κωδικού του
		\item Ακολουθεί τον σύνδεσμο και εισάγει τον νέο κωδικού του λογαριασμού του
		\item Συνδέεται στην πλατφόρμα με χρήση του νέου κωδικού του
		\vspace{-0.7cm}
	\end{enumerate} \\ \hline
	Εναλλακτικό σενάριο: & Αν ο χρήστης εισάγει λάθος όνομα χρήστη στο βήμα 3, δε θα λάβει το email επαναφοράς του κωδικού του. \\ \hline
\end{longtable}

%
% Υπενθύμιση Ονόματος Χρήστη
%
\begin{longtable}{|p{0.2\linewidth}|p{0.7\linewidth}|} 
	\caption{Υπενθύμιση Ονόματος Χρήστη} \label{tab:use-case-forgot-username} \\
	\hline
	\endfirsthead
	\caption{Υπενθύμιση Ονόματος Χρήστη (συνέχεια)} \\ 
	\endhead \endfoot 
	Περιγραφή: & Ο τρόπος με τον οποίο ο χρήστης ζητά email υπενθύμισης για το όνομα χρήστη του λογαριασμού του \\ \hline
	Χειριστές: & Γενικός Διαχειριστής Πλατφόρμας, Διαχειριστής Εργαστηρίου \\ \hline
	Προϋποθέσεις: & Καμία \\ \hline
	Βασική ροή: & 
	\begin{enumerate}
		\vspace{-1cm}
		\addtolength{\itemindent}{-0.4cm}
		\item Ο χρήστης έχει ξεχάσει το όνομα χρήστη του λογαριασμού του
		\item Ο χρήστης περιηγείται στο σημείο υπενθύμισης ονόματος χρήστη
		\item Εισάγει το email του λογαριασμού του
		\item Λαμβάνει στο email του υπενθύμιση του ονόματος χρήστη του
		\item Ακολουθεί τον σύνδεσμο και εισάγει τον νέο κωδικού του λογαριασμού του
		\item Συνδέεται στην πλατφόρμα με χρήση των στοιχείων σύνδεσης του
		\vspace{-0.7cm}
	\end{enumerate} \\ \hline
	Εναλλακτικό σενάριο: & Αν ο χρήστης εισάγει λάθος email στο βήμα 3, δε θα λάβει το email υπενθύμισης του ονόματος χρήστη του. \\ \hline
\end{longtable}

%
% Αποσύνδεση από το σύστημα
%
\begin{longtable}{|p{0.2\linewidth}|p{0.7\linewidth}|} 
	\caption{Αποσύνδεση από το σύστημα} \label{tab:use-case-logout} \\
	\hline
	\endfirsthead
	\caption{Αποσύνδεση από το σύστημα (συνέχεια)} \\ 
	\endhead \endfoot 
	Περιγραφή: & Ο τρόπος με τον οποίο ο χρήστης αποσυνδέεται από την πλατφόρμα \\ \hline
	Χειριστές: & Γενικός Διαχειριστής Πλατφόρμας, Διαχειριστής Εργαστηρίου \\ \hline
	Προϋποθέσεις: & Καμία \\ \hline
	Βασική ροή: & 
	\begin{enumerate}
		\vspace{-1cm}
		\addtolength{\itemindent}{-0.4cm}
		\item Επιλογή της Αποσύνδεσης από το κεντρικό μενού.
		\item Ο χρήστης αποσυνδέεται από το σύστημα και διαγράφεται το cookie αυθεντικοποίησης χρήστη της εφαρμογής από τον περιηγητή ιστού του
		\vspace{-0.7cm}
	\end{enumerate} \\ \hline
	Εναλλακτικό σενάριο: & \\ \hline
\end{longtable}

%
% Εισαγωγή υπολογιστή
%
\begin{longtable}{|p{0.2\linewidth}|p{0.7\linewidth}|} 
	\caption{Εισαγωγή υπολογιστή} \label{tab:use-case-add-computer} \\
	\hline
	\endfirsthead
	\caption{Εισαγωγή υπολογιστή (συνέχεια)} \\ 
	\endhead \endfoot 
	Περιγραφή: & Ο τρόπος με τον οποίο ο διαχειριστής προσθέτει έναν υπολογιστή στο σύστημα \\ \hline
	Χειριστές: & Γενικός Διαχειριστής Πλατφόρμας \\ \hline
	Προϋποθέσεις: & Πρέπει να έχει ρυθμιστεί ο DHCP server του εργαστηρίου στο οποίο βρίσκεται ο υπολογιστής, ώστε να παρέχει τη διεύθυνση του συστήματος για δικτυακή εκκίνηση των υπολογιστών \\ \hline
	Βασική ροή: & 
	\begin{enumerate}
		\vspace{-1cm}
		\addtolength{\itemindent}{-0.4cm}
		\item Ο υπολογιστής εκκινείται και τα στοιχεία του υπολογιστή (uuid της μητρικής πλακέτας και mac της διεπαφής δικτύου) καταγράφονται στο σύστημα και εμφανίζονται στην οθόνη του υπολογιστή
		\item Ο διαχειριστής βλέπει την νέα εγγραφή στο σύστημα, κάνει την επαλήθευση της ταυτότητας του υπολογιστή, συγκρίνοντας τα στοιχεία που βλέπει στο σύστημα με αυτά που εμφανίζονται στην οθόνη του υπολογιστή και αποφασίζει αν θα κάνει δεκτό ή όχι τον υπολογιστή
		\item Ο διαχειριστής κάνει δεκτό τον υπολογιστή και συμπληρώνει τα απαραίτητα στοιχεία για αυτόν (το όνομά του, το εργαστήριο στο οποίο βρίσκεται, τις ομάδες υπολογιστών στις οποίες ανήκει)
		\item Ο υπολογιστής είναι έτοιμος να εκκινηθεί μέσω του συστήματος και μπορούν να διαχειριστούν τα στοιχεία του, τόσο οι γενικοί διαχειριστές του συστήματος, όσο και οι διαχειριστές του εργαστηρίου στο οποίο ανήκει
		\vspace{-0.7cm}
	\end{enumerate} \\ \hline
	Εναλλακτικό σενάριο: & Ο διαχειριστής αποφασίζει στο βήμα 3 να μην κάνει δεκτό τον υπολογιστή, οπότε τα στοιχεία του διαγράφονται από το σύστημα \\ \hline
\end{longtable}

\FloatBarrier

\section{Σχεδιασμός Βάσης Δεδομένων}
Στους πίνακες που ακολουθούν παρουσιάζονται οι πίνακες της βάσης δεδομένων του συστήματος, καθώς και τα πεδία του κάθε πίνακα. Επίσης, ως σχόλιο σε κάθε πεδίο, περιγράφεται η χρησιμότητά του.

\FloatBarrier
\input{Chapters/3-4-Database-Schema}
\FloatBarrier

\section{Επεξήγηση Αρχείων}

\section{Ασφάλεια Συστήματος}

\section{Σύνοψη Κεφαλαίου 3}
