\chapter{Θεωρητικό Υπόβαθρο}
Στο κεφάλαιο αυτό, αναλύονται λεπτομερώς όλα τα είδη τεχνολογιών που χρησιμοποιήθηκαν για την υλοποίηση της διαδικτυακής εφαρμογής που κατασκευάστηκε για την εκπόνηση της διπλωματικής εργασίας.

\section{Διαδίκτυο}
Το Διαδίκτυο αποτελείται από μια σειρά δικτύων που συνδέουν συσκευές σε όλο τον κόσμο. Αυτά τα δίκτυα χρησιμοποιούν τηλεφωνικές γραμμές για τη σύνδεση των διαφόρων συσκευών. Οι χρήστες χρησιμοποιούν το Διαδίκτυο από παρόχους υπηρεσιών Διαδικτύου. Η κινητή ευρυζωνικότητα και το Wi-Fi έχουν γίνει κοινός τόπος στον 21ο αιώνα, επιτρέποντας την ασύρματη σύνδεση. Η σουίτα πρωτοκόλλων TCP/IP είναι ένα ευρέως χρησιμοποιούμενο σύνολο πρωτοκόλλων δικτύωσης που επιτρέπουν την επικοινωνία μεταξύ διαφορετικών δικτύων σε όλο τον κόσμο. Το Διαδίκτυο είναι ένα καινοτόμο σύστημα που έχει φέρει επανάσταση στο εμπόριο, επιτρέποντας σε ανθρώπους από διαφορετικά μέρη του κόσμου να συνδεθούν \cite{internet_britannica}.

Το Διαδίκτυο μπορεί να χρησιμοποιηθεί για σχεδόν κάθε σκοπό που εξαρτάται από την πληροφορία, και όποιος έχει την δυνατότητα να συνδεθεί σε ένα από τα δίκτυα που το συνθέτουν, έχει πρόσβαση σε αυτό. Υποστηρίζει την ανθρώπινη επικοινωνία μέσω των κοινωνικών μέσων, του ηλεκτρονικού ταχυδρομείου (e-mail), των "chat rooms", και επιτρέπει στους ανθρώπους να συνεργάζονται είτε σύγχρονα, είτε ασύγχρονα σε πολλά διαφορετικά μέρη του κόσμου. Υποστηρίζει την πρόσβαση σε ψηφιακές πληροφορίες μέσω πολλών εφαρμογών, συμπεριλαμβανομένου του Παγκόσμιου Ιστού. Το Διαδίκτυο έχει διαδραματίσει σημαντικό ρόλο στην ανάπτυξη ενός μεγάλου και αυξανόμενου αριθμού "ηλεκτρονικών επιχειρήσεων" (συμπεριλαμβανομένων και των θυγατρικών των παραδοσιακών εταιρειών) που χρησιμοποιούν το Διαδίκτυο για το μεγαλύτερο μέρος των πωλήσεων και των υπηρεσιών τους.

\subsection{Παγκόσμιος Ιστός}
Αν και μερικές φορές χρησιμοποιούνται εναλλακτικά, οι όροι Διαδίκτυο και Παγκόσμιος Ιστός δεν σημαίνουν το ίδιο. Ο Παγκόσμιος Ιστός είναι μία από τις κύριες υπηρεσίες που παρέχονται μέσω του Διαδικτύου, ενώ ο όρος "Διαδίκτυο" αναφέρεται σε ολόκληρο το παγκόσμιο σύστημα επικοινωνίας, που περιλαμβάνει το υλικό και την υποδομή. Πιο επίσημα, ο Παγκόσμιος Ιστός μπορεί να οριστεί ως ένα σύστημα τεχνικοκοινωνικής αλληλεπίδρασης. Ένα σύστημα που βελτιώνει την ανθρώπινη διάνοια, την επικοινωνία και τη συνεργασία αναφέρεται ως τεχνοκοινωνικό σύστημα. Οι πληροφορίες που δημιουργούνται στον Παγκόσμιο Ιστό από τους χρήστες του μπορούν να αποθηκεύονται στο Διαδίκτυο. Υπάρχουν διάφοροι ιστότοποι στον Παγκόσμιο Ιστό, καθένας από τους οποίους έχει τη δική του διεύθυνση URL (Uniform Resource Locator). Η διεύθυνση URL έχει τη μορφή http://www.example.com. Το http συνιστά τη βάση δεδομένων επικοινωνίας για τον Παγκόσμιο Ιστό, το www πρόκειται για το World Wide Web, το οποίο είναι ένα πληροφοριακό σύστημα όπου τα έγγραφα και άλλοι δικτυακοί τόποι αναγνωρίζονται από τις διευθύνσεις URL τους, το example δηλώνει το όνομα του δικτυακού τόπου (domain name) και το com υποδεικνύει την περιοχή στην οποία ανήκει ο δικτυακός τόπος ή τον τύπο του δικτυακού τόπου \cite{aghaei2012evolution}.

\subsection{Εφαρμογή Ιστού}
Μια Εφαρμογή Web (Ιστού) συνιστά μια εφαρμογή που διανέμεται μέσω του Διαδικτύου με τη χρήση ενός προγράμματος περιήγησης και διατηρείται σε έναν απομακρυσμένο διακομιστή. Εξ ορισμού, οι υπηρεσίες διαδικτύου είναι Εφαρμογές Ιστού και πολλοί ιστότοποι -αν και όχι όλοι- περιέχουν τέτοιες εφαρμογές. Οι Εφαρμογές Ιστού μπορούν να δημιουργηθούν για ένα ευρύ φάσμα σκοπών και να χρησιμοποιηθούν από ανθρώπους ή οργανισμούς για πολλά διαφορετικά πράγματα. Το webmail, οι ηλεκτρονικές αριθμομηχανές και οι ιστότοποι ηλεκτρονικών αγορών είναι μερικά παραδείγματα συχνά χρησιμοποιούμενων Εφαρμογών Ιστού. Αν και η πλειονότητα των εφαρμογών ιστού είναι προσβάσιμες από οποιοδήποτε πρόγραμμα περιήγησης, ορισμένες είναι προσβάσιμες μόνο από ένα συγκεκριμένο πρόγραμμα περιήγησης \cite{Web_Apps}.

\section{Τεχνικές Προγραμματισμού}

\subsection{Αντικειμενοστρεφής Προγραμματισμός}
Ο αντικειμενοστρεφής προγραμματισμός (Object Oriented Programming / OOP), αποτελεί πρότυπο στην ανάπτυξη εφαρμογών, σε οποιαδήποτε γλώσσα. Είναι ένα μοντέλο προγραμματισμού που βοηθά στην οργάνωση του σχεδιασμού λογισμικού γύρω από δεδομένα ή αντικείμενα. Το μοντέλο αυτό συμβάλλει στη βελτίωση της σαφήνειας και της οργάνωσης του κώδικα, καθιστώντας τον ευκολότερο στην κατανόηση και τη συντήρησή του. Ένα αντικείμενο μπορεί να οριστεί ως ένα πεδίο δεδομένων που έχει μοναδικά χαρακτηριστικά και συμπεριφορά.

\subsection{Αρχιτεκτονική MVC}

\subsection{RESTful API}

\section{Γλώσσες Προγραμματισμού Ιστού}

\subsection{HTML}
Η HTML (Hyper Text Markup Language - Γλώσσα Σήμανσης Υπερκειμένου) αποτελεί την βασική γλώσσα για την κατασκευή ενός ιστοχώρου. Δεν θεωρείται γλώσσα προγραμματισμού, αλλά αποτελεί μία περιγραφική γλώσσα (markup language) η οποία περιέχει οδηγίες προς τους web browsers. Οι web browsers αφού διαβάσουν τις οδηγίες αυτές, τις μεταφράζουν στις κατάλληλες εντολές για να δημιουργηθεί το οπτικό περιεχόμενο που θα παρουσιαστεί στον χρήστη. Η παραπάνω διαδικασία επιτυγχάνεται με την βοήθεια των HTML elements, τα οποία οριοθετούνται από ετικέτες (tags). Οι ετικέτες είναι γράμματα ή λέξεις, τα οποία περικλείουν γωνιώδεις αγκύλες και υπάρχουν ανά ζεύγη. Χωρίζονται σε tags έναρξης, τα οποία σηματοδοτούν την έναρξη μιας εντολής, και σε tags λήξης, τα οποία σηματοδοτούν την λήξη της εντολής, με μερικές εξαιρέσεις. Οι ετικέτες λήξης διαχωρίζονται από τις ετικέτες έναρξης με μία πλάγια γραμμή '/'. Ένα βασικό παράδειγμα ετικέτας είναι το: <html>...</html>, ανάμεσα στις ετικέτες περικλείεται κείμενο ή ακόμη και άλλες εσωτερικές ετικέτες (εμφωλευμένα tags). Ορισμένες βασικές ετικέτες παρατίθενται στον πίνακα \ref{tbl:html_basic_elements}.

\begin{table}[h]
	\caption{Βασικά στοιχεία HTML}
	\label{tbl:html_basic_elements}
	\begin{tabular}{|p{0.2\linewidth}|p{0.7\linewidth}|}
		\hline
		\textless{}html\textgreater{}...\textless{}/html\textgreater{} & Η αρχή και το τέλος του HTML αρχείου \\ \hline
		\textless{}!DOCTYPE\textgreater{} & Η οδηγία που καθορίζει την έκδοση της HTML που χρησιμοποιείται \\ \hline
		\textless{}head\textgreater{}...\textless{}/head\textgreater{} & Οι σχετικές πληροφορίες με το έγγραφο (όπως η γλώσσα, η κωδικοποίηση και τα μεταδεδομένα) \\ \hline
		\textless{}title\textgreater{}...\textless{}/title\textgreater{} & Ο τίτλος του αρχείου \\ \hline
		\textless{}body\textgreater{}...\textless{}/body\textgreater{} & Τα οπτικά στοιχεία του αρχείου \\ \hline
		\textless{}div\textgreater{}...\textless{}/div\textgreater{} & Ομαδοποίηση στοιχείων εντός ετικέτας \\ \hline
		\textless{}input\textgreater{}...\textless{}/input\textgreater{} & Ορισμός πεδίου εισαγωγής \\ \hline
		\textless{}!--...--\textgreater{} & Ορισμός σχολίων \\ \hline
		\textless{}form\textgreater{}...\textless{}/form\textgreater{} & Ορισμός φόρμας \\ \hline
		\textless{}button\textgreater{}...\textless{}/button\textgreater{} & Ορισμός κουμπιού \\ \hline
		\textless{}a\textgreater{}...\textless{}/a\textgreater{} & Ορισμός υπερσυνδέσμου \\ \hline
	\end{tabular}
\end{table}

\subsection{CSS}
Η CSS αποτελεί μία γλώσσα επικαλυπτόμενων στυλ μορφοποίησης (Cascading Style Sheets) και όπως η HTML δεν θεωρείται καθαρή γλώσσα προγραμματισμού. Συντελεί στον διαχωρισμό των εντολών εμφάνισης από τις εντολές του περιεχομένου της ιστοσελίδας. Χρησιμοποιείται για τη μορφοποίηση οποιασδήποτε ετικέτας HTLM και τη δημιουργία ενός αποτελέσματος πιο όμορφου οπτικά για τον χρήστη. Η σύνταξη μιας εντολής στην CSS αποτελείται από τρία κύρια στοιχεία: το στοιχείο που θα τροποποιηθεί, τις ιδιότητες του στοιχείου που θα επηρεαστούν και η νέα τιμή των ιδιοτήτων. Στον πίνακα \ref{tbl:css_basic_elements} αναφέρεται ένα παράδειγμα του τρόπου σύνταξης μιας εντολής στην γλώσσα CSS.

\begin{table}[h]
	\caption{Στοιχεία εντολής CSS}
	\label{tbl:css_basic_elements}
	\centering
	\begin{tabular}{|l|}
		\hline
		\begin{tabular}[c]{@{}l@{}}
			p \{ \\ \quad 
			font-family: Calibri; \\ \quad 
			color: \#1B1811; \\ \quad 
			text-align: left; \\ \}
		\end{tabular} \\ \hline
	\end{tabular}
\end{table}

Όπως φαίνεται στον πίνακα \ref{tbl:css_basic_elements}, το στοιχείο p αποτελεί την ετικέτα της παραγράφου στην HTML και ονομάζεται επιλογέας (css selector). Τα font-family, color και text-align είναι οι ιδιότητες της εντολής CSS και δείχνουν τα στοιχεία του επιλογέα τα οποία θα τροποποιηθούν. Ο διαχωρισμός των ιδιοτήτων από τις νέες τιμές που θα λάβουν σηματοδοτείται με την άνω και κάτω τελεία (:), στης οποίας το αριστερό μέρος βρίσκονται οι ιδιότητες και στο δεξί οι νέες τιμές τους.

\subsection{JavaScript \& Ajax}
Η JavaScript είναι μια γλώσσα προγραμματισμού σεναρίου (scripting language), η οποία εκτελείται από τους περιηγητές ιστών (web browsers) χρησιμοποιώντας έναν σχετικό διερμηνευτή (Interpreter). Στόχο της αποτελεί η βελτίωση της εμπειρίας χρήσης. Με την JavaScript υπάρχει η δυνατότητα προσθήκης ή αφαίρεσης HTML στοιχείων και CSS κανόνων καθώς και η τροποποίηση ιδιοτήτων HTLM στοιχείων με την μεταβολή στις τιμές τους. Αν και η JavaScript αποτελεί μια γλώσσα προγραμματισμού με το Client-Side χαρακτηριστικό, τον τελευταίο καιρό γίνεται χρήση της και από την πλευρά των υπολογιστών εξυπηρετητών.

Η Ajax (Asynchronous JavaScript and XML) αποτελείται από τον συνδυασμό των τεχνολογιών JavaScript και XML. Είναι μία τεχνολογία, η οποία προσδίδει διαδραστικές δυνατότητες σε μία ιστοσελίδα. Μέσω της Ajax γίνεται εφικτή η ανανέωση μέρους της ιστοσελίδας (στο παρασκήνιο θα γίνει επικοινωνία της τεχνολογίας με τον server, ο οποίος θα λάβει τα δεδομένα που ζητήθηκαν και με τη σειρά του θα τα εμφανίσει στον χρήστη), χωρίς να χρειαστεί να γίνει ανανέωση (refresh) ολόκληρης της ιστοσελίδας.

\subsection{PHP}
Η PHP ορίζεται ως μία αντικειμενοστρεφής γλώσσα προγραμματισμού γενικής χρήσης (παλαιότερα αποτελούσε γλώσσα προγραμματισμού σεναρίου), η οποία χρησιμοποιείται κυρίως για την ανάπτυξη διαδικτυακών εφαρμογών (web apps), δηλαδή αποτελεί την κατάλληλη γλώσσα για την δημιουργία ιστοχώρων με δυναμικό περιεχόμενο. Διατίθεται σε δύο μορφές, σε μορφή πηγαίου κώδικα και σε δυαδική μορφή και οι δύο αυτές μορφές έχουν ελεύθερη πρόσβαση. Η PHP χρησιμοποιείται για τον χειρισμό λειτουργιών και εργασιών τις οποίες θα υλοποιήσει και όχι για την οπτική διαμόρφωση μίας ιστοσελίδας.

Επομένως, ο χρήστης λαμβάνει τα αποτελέσματα του σεναρίου (στον browser ως απλές σελίδες HTLM) και όχι τον κώδικα, ο οποίος εκτελείται στον server με τη χρήση του αντίστοιχου διερμηνευτή (interpreter) της κάθε γλώσσας. Ο διερμηνευτής, αφού διαβάσει τον κώδικα, εκτελεί τις δηλώσεις της γλώσσας ανά βήμα και τις μετατρέπει σε εκτελέσιμο κώδικα για το υπολογιστικό σύστημα.

Αναλυτικότερα, η PHP έχει την ικανότητα να δημιουργήσει, γράψει, διαβάσει, ανοίξει, κλείσει και διαγράψει αρχεία στη Βάση Δεδομένων. Υποστηρίζει ένα ευρύ φάσμα Βάσεων Δεδομένων, καθώς επίσης είναι δωρεάν και αρκετά εύκολη στην εκμάθηση, ενώ ταυτόχρονα είναι συμβατή και μπορεί να τρέξει σε οποιαδήποτε πλατφόρμα, όπως για παράδειγμα Windows, Mac OS X, Linux, Unix κ.α. Ένα αρχείο PHP (έγγραφο κειμένου αποθηκευμένο με την κατάληξη .php) είναι δυνατό να περιέχει εκτός από τον κώδικα PHP και κώδικα HTLM, CSS και JavaScript.

Ορισμένες βασικές συναρτήσεις της PHP παρατίθενται στον πίνακα \ref{tbl:php_basic_functions}.

\begin{table}[h]
	\caption{Βασικές συναρτήσεις της PHP}
	\label{tbl:php_basic_functions}
	\begin{tabular}{|p{0.2\linewidth}|p{0.75\linewidth}|}
		\hline
		htmlspecialchars() & Μετατροπή ειδικών χαρακτήρων σε οντότητες HTML \\ \hline
		explode() & Μετατροπή μιας συμβολοσειράς σε πίνακα με χρήση ενός διαχωριστικού χαρακτήρα \\ \hline
		rand() & Επιστροφή ενός τυχαίου αριθμού \\ \hline
		str\_replace() & Εύρεση και αντικατάσταση ενός μοτίβου σε μια συμβολοσειρά \\ \hline
		date() & Επιστροφή μιας μορφοποιημένης ημερομηνίας \\ \hline
		strlen() & Επιστροφή του μήκους μιας συμβολοσειράς \\ \hline
		count() & Επιστροφή του αριθμού των στοιχείων ενός πίνακα \\ \hline
		array\_unique() & Απαλοιφή διπλότυπων στοιχείων ενός πίνακα \\ \hline
		print\_r() & Εκτύπωση μεταβλητής \\ \hline
		echo & Εκτύπωση μίας ή περισσότερων συμβολοσειρών \\ \hline
	\end{tabular}
\end{table}

\subsection{SQL}
Η Structured Query Language (SQL) \cite{Loshin_2022}, μία τυποποιημένη γλώσσα προγραμματισμού, χρησιμοποιείται για τη διαχείριση σχεσιακών βάσεων δεδομένων και την εκτέλεση διαφόρων πράξεων στα δεδομένα που περιέχουν. Από την ίδρυσή της τη δεκαετία του 1970, η SQL χρησιμοποιείται ευρέως από διαχειριστές βάσεων δεδομένων καθώς και από προγραμματιστές που δημιουργούν σενάρια για την ολοκλήρωση δεδομένων και από αναλυτές δεδομένων που κατασκευάζουν και εκτελούν αναλυτικά ερωτήματα.

Με τη βοήθεια της ευέλικτης γλώσσας SQL, οι χρήστες μπορούν να λαμβάνουν, να αποθηκεύουν, να επεξεργάζονται και να διαγράφουν δεδομένα, καθώς και να δημιουργούν, να τροποποιούν και να αφαιρούν αντικείμενα βάσεων δεδομένων (όπως πίνακες, στήλες, διαδικασίες και χρήστες), καθώς και να δίνουν και να ανακαλούν προνόμια χρηστών και να ομαδοποιούν δηλώσεις σε συναλλαγές. Μια δήλωση που ζητά δεδομένα από τη βάση δεδομένων αναφέρεται ως ερώτημα και μια εντολή SQL είναι γνωστή ως δήλωση. Στις εντολές SQL περιλαμβάνονται προγνωστικά (όπως LIKE, BETWEEN, EXISTS), τελεστές (όπως AND, OR, NOT), ποσοδείκτες (όπως ANY, ALL, UNION), συναρτήσεις (όπως COUNT, SUM, AVG) και προτάσεις (όπως SELECT, FROM, WHERE). Όταν δεν είναι σημαντικό να γίνει διάκριση μεταξύ, για παράδειγμα, των ρητρών και των κατηγορημάτων, αναφερόμαστε σε αυτά ως έννοιες συλλογικά.
Η γλώσσα χειρισμού δεδομένων (DML, π.χ. SELECT, INSERT, UPDATE, DELETE) και η γλώσσα ορισμού δεδομένων (DDL, π.χ. CREATE, ALTER, DROP) είναι οι δύο υπογλώσσες της SQL που χρησιμοποιούνται συχνότερα \cite{taipalus2020sql}.
 
Τα δεδομένα αποθηκεύονται, ανακτώνται και αναλύονται με τη χρήση λογισμικού που ονομάζεται σύστημα διαχείρισης βάσεων δεδομένων (DBMS). Οι χρήστες μπορούν να δημιουργούν, να διαβάζουν, να ενημερώνουν και να διαγράφουν δεδομένα σε βάσεις δεδομένων χρησιμοποιώντας ένα DBMS, το οποίο λειτουργεί ως διεπαφή μεταξύ αυτών και των βάσεων δεδομένων. Η ασφάλεια των δεδομένων, η ακεραιότητα των δεδομένων, η ταυτόχρονη χρήση και οι τυποποιημένες πρακτικές διαχείρισης δεδομένων βοηθούνται από αυτό.

Τα μοντέλα δεδομένων, οι κατανομές βάσεων δεδομένων, ο αριθμός των χρηστών και άλλοι παράγοντες μπορούν να χρησιμοποιηθούν για την κατηγοριοποίηση των συστημάτων διαχείρισης βάσεων δεδομένων. Οι σχεσιακές, κατανεμημένες, ιεραρχικές, αντικειμενοστραφείς και δικτυακές μορφές λογισμικού DBMS είναι οι πιο δημοφιλείς \cite{DBMS}.

Στα συστήματα διαχείρισης σχεσιακών βάσεων δεδομένων (RDBMS), η SQL χρησιμοποιείται για τις ακόλουθες εργασίες: αλλαγή των δομών των πινάκων και των δεικτών της βάσης δεδομένων, προσθήκη, ενημέρωση και διαγραφή γραμμών δεδομένων και ανάκτηση υποσυνόλων πληροφοριών. Οι πληροφορίες αυτές μπορούν να χρησιμοποιηθούν για την επεξεργασία συναλλαγών, εφαρμογές ανάλυσης και άλλες εφαρμογές που απαιτούν αλληλεπίδραση με μια σχεσιακή βάση δεδομένων.

\section{Πλαίσια και Βιβλιοθήκες Προγραμματισμού Ιστού}

\subsection{Bootstrap}
Το Bootstrap αποτελεί μια βιβλιοθήκη HTML, CSS και JS με έμφαση στην απλοποίηση της διαδικασίας ανάπτυξης ιστοσελίδων (σε αντίθεση με τις εφαρμογές ιστού). Αρχικά σχεδιασμένο από το Twitter, πλέον αναπτύσσεται από την ομάδα του Bootstrap στο GitHub, για front-end ανάπτυξη ιστοσελίδων που δίνει προτεραιότητα στις κινητές συσκευές, το Bootstrap είναι ένα δωρεάν και ανοιχτού κώδικα πλαίσιο CSS. Περιλαμβάνει πρότυπα σχεδιασμού για τυπογραφία, φόρμες, κουμπιά, πλοήγηση και άλλα στοιχεία διεπαφής σε HTML, CSS και (προαιρετικά) JavaScript \cite{Bootstrap}. 

Ο κύριος στόχος της προσθήκης του σε ένα έργο ιστού είναι η εφαρμογή των επιλογών χρώματος, μεγέθους, γραμματοσειράς και διάταξης του Bootstrap σε αυτό το έργο. Ως εκ τούτου, ο κύριος καθοριστικός παράγοντας είναι το κατά πόσον αυτές οι επιλογές αρέσουν στους υπεύθυνους προγραμματιστές. Όλα τα στοιχεία HTML έχουν βασικές δηλώσεις στυλ μόλις το Bootstrap εισαχθεί σε ένα έργο. 

Ως αποτέλεσμα, τα κείμενα, οι πίνακες και τα στοιχεία φόρμας εμφανίζονται με συνέπεια σε όλα τα προγράμματα περιήγησης ιστού. Προκειμένου να εξατομικεύσουν περαιτέρω την εμφάνιση του περιεχομένου τους, οι προγραμματιστές μπορούν να κάνουν χρήση των κλάσεων CSS που ορίζονται στο Bootstrap. Για παράδειγμα, το Bootstrap προσφέρει ενσωματωμένη υποστήριξη για ανοιχτόχρωμους και σκούρους πίνακες, επικεφαλίδες σελίδων, μεγαλύτερα pull quotes και κείμενο με υπογράμμιση \cite{bootstrap_2}.

Επιπλέον, το Bootstrap περιλαμβάνει μια σειρά από στοιχεία JavaScript που μπορούν να χρησιμοποιηθούν ανεξάρτητα από άλλα πλαίσια, όπως το jQuery. Προσφέρουν επιπλέον στοιχεία UI, όπως πλαίσια διαλόγου, tooltips, μπάρες προόδου, drop-down μενού και καρουσέλ \cite{gaikwad2019review}. Κάθε στοιχείο Bootstrap αποτελείται από ένα πλαίσιο HTML, δηλώσεις CSS και περιστασιακά από συμπληρωματικό κώδικα JavaScript. Ταυτόχρονα, αυξάνουν τη λειτουργικότητα μερικών ήδη υπαρχόντων στοιχείων διεπαφής, όπως η λειτουργία αυτόματης συμπλήρωσης για τα πεδία εισαγωγής.

\subsection{CodeIgniter}


\section{Τεχνολογίες Ανάπτυξης Λογισμικού}

\subsection{Jetbrains PhpStorm}
Κατασκευασμένο από την τσεχική εταιρεία JetBrains, το PhpStorm είναι ένα ιδιόκτητο, διαπλατφορμικό IDE (ολοκληρωμένο περιβάλλον ανάπτυξης) για την PHP \cite{grigorev2014string}. Με on-the-fly ανάλυση κώδικα, πρόληψη σφαλμάτων και αυτοματοποιημένες αναδιαμορφώσεις για κώδικα PHP και JavaScript, το PhpStorm προσφέρει έναν επεξεργαστή για PHP, HTML και JavaScript. Η PHP υποστηρίζεται από τη συμπλήρωση κώδικα του PhpStorm, η οποία υποστηρίζει επίσης namespaces, closures, traits, generators, coroutines, τη λέξη-κλειδί finally, λίστες σε foreach και σύνταξη σύντομων πινάκων. Περιλαμβάνεται ένας πλήρης επεξεργαστής SQL με επεξεργάσιμα αποτελέσματα ερωτημάτων.

Για τη συγγραφή του PhpStorm χρησιμοποιείται η Java. Εγκαθιστώντας plugins που έχουν κατασκευαστεί για το PhpStorm ή δημιουργώντας τα δικά τους, οι χρήστες μπορούν να επεκτείνουν τη λειτουργικότητα του IDE. Επιπλέον, το λογισμικό αλληλεπιδρά με εξωτερικούς πόρους όπως το XDebug. Το PhpStorm περιέχει όλες τις δυνατότητες που υπάρχουν στο WebStorm. Τα πρόσθετα JavaScript είναι ήδη προεγκατεστημένα στο WebStorm \cite{PhpStorm_JetBrains}.

\subsection{Git}
Το Global Information Tracker (Git) είναι ένα σύστημα ελέγχου εκδόσεων (VCS) διαθέσιμο σε όλες τις κύριες πλατφόρμες ανάπτυξης μέσω μιας άδειας χρήσης ελεύθερου λογισμικού \cite{spinellis2012git}. Ορισμένα χαρακτηριστικά που μπορεί να βρεθούν στο git είναι η διευκόλυνση της κατανεμημένης ανάπτυξης μέσω πολλαπλών προγραμματιστών σε πολλαπλές τοποθεσίες, γεγονός που οδηγεί στην κλίμακα για τον χειρισμό των χιλιάδων προγραμματιστών που εργάζονται στα ίδια ή διαφορετικά μέρη ενός έργου. Η γρήγορη και αποδοτική απόδοση με τη χρήση τεχνικών "delta" και συμπίεσης, η διατήρηση της ακεραιότητας των δεδομένων με την κρυπτογραφική συνάρτηση κατακερματισμού (SHA1) και η επιβολή της λογοδοσίας. Επίσης, υποστηρίζει και ενθαρρύνει τη διακλαδισμένη και σε άλλες περιπτώσεις τη συγχωνευμένη ανάπτυξη. Μία ακόμη ιδιότητα του git είναι πως διαθέτει πλήρη αποθετήρια. Ένα αποθετήριο αποτελείται από πολλούς φακέλους και αρχεία τα οποία είναι προσβάσιμα από οποιονδήποτε έχει πρόσβαση στο ιδιωτικό αποθετήριο \cite{loeliger2012version}. 

Το Git διαθέτει έναν αριθμό από Graphical User Interfaces (GUI) πελάτες και ένας από αυτούς είναι το Gitkraken \cite{gitkrakenwhatisit}, το οποίο αποτελεί ένα γραφικό περιβάλλον διαχείρισης git αποθετηρίων. Χρησιμοποιείται για την εύκολη και αποτελεσματική διαχείριση, την δημιουργία, αλλά και την συγχώνευση των διακλαδώσεων. Διαθέτει την δυνατότητα να ενσωματώνεται με τον λογαριασμό GitHub ή Bitbucket ενός χρήστη, καθώς επίσης μπορεί να προσαρμοστεί εύκολα στον χώρο εργασίας του χρήστη.

Το GitHub είναι μία διαδικτυακή πλατφόρμα συνεργατικής φιλοξενίας κώδικα ανάπτυξης λογισμικού, που βασίζεται στο σύστημα ελέγχου εκδόσεων git. Μέσω του GitHub, οι προγραμματιστές έχουν την δυνατότητα να αντιγράψουν ένα αποθετήριο στον λογαριασμό τους (fork) και αφού πραγματοποιήσουν τις αλλαγές τους, τους δίνεται η δυνατότητα να προτείνουν  να ενσωματωθούν στο αρχικό αποθετήριο από τους συντηρητές του (pull request και merge). Οι προγραμματιστές μπορούν επίσης να ακολουθήσουν άλλους χρήστες, σχηματίζοντας έτσι ένα κύκλωμα ανταλλαγής πληροφοριών, όπου κάθε προγραμματιστής ενημερώνεται για εξελίξεις σχετικές με τα ενδιαφέροντα του \cite{kalliamvakou2014promises}. Εκτός από τις λειτουργίες του Git, το GitHub προσφέρει και μερικά επιπρόσθετα χαρακτηριστικά. Βοηθά στη διαχείριση της πρόσβασης και της συνεργασίας για έργα, παρέχοντας χαρακτηριστικά όπως είναι η παρακολούθηση σφαλμάτων, η διαχείριση των διαφόρων εργασιών και τα αιτήματα χαρακτηριστικών.
