\chapter{Θεωρητικό Υπόβαθρο}

\section{Προγραμματισμός διαδικτύου}

\section{HTML}
Η HTML (Hyper Text Markup Language - Γλώσσα Σήμανσης Υπερκειμένου) αποτελεί την βασική γλώσσα για την κατασκευή ενός ιστοχώρου. Δεν θεωρείται γλώσσα προγραμματισμού, αλλά αποτελεί μία περιγραφική γλώσσα (markup language) η οποία περιέχει οδηγίες προς τους web browsers. Οι web browsers αφού διαβάσουν τις οδηγίες αυτές, τις μεταφράζουν στις κατάλληλες εντολές για να δημιουργηθεί το οπτικό περιεχόμενο που θα παρουσιαστεί στον χρήστη. Η παραπάνω διαδικασία επιτυγχάνεται με την βοήθεια των HTML elements, τα οποία οριοθετούνται από ετικέτες (tags). Οι ετικέτες είναι γράμματα ή λέξεις, τα οποία περικλείουν γωνιώδεις αγκύλες και υπάρχουν ανά ζεύγη. Χωρίζονται σε tags έναρξης, τα οποία σηματοδωτούν την έναρξη μιας εντολής, και σε tags λήξης, τα οποία σηματοδωτούν την λήξη της εντολής, με μερικές εξαιρέσεις. Οι ετικέτες λήξης διαχωρίζονται από τις ετικέτες έναρξης με μία πλάγια γραμμή '/'. Ένα βασικό παράδειγμα ετικέτας είναι το: <html>...</html>, ανάμεσα στις ετικέτες περικλείεται κείμενο ή ακόμη και άλλες εσωτερικές ετικέτες (εμφωλευμένα tags). Ορισμένες βασικές ετικέτες παρατίθενται στον πίνακα \ref{tbl:html_basic_elements}.

\begin{table}[h]
\caption{Βασικά στοιχεία HTML}
\label{tbl:html_basic_elements}
\begin{tabular}{|p{0.3\linewidth}|p{0.65\linewidth}|}
\hline
\textless{}html\textgreater{}...\textless{}/html\textgreater{}     & Η αρχή και το τέλος του HTML αρχείου                                                      \\ \hline
\textless{}!DOCTYPE\textgreater{}                                  & Η οδηγία που καθορίζει την έκδοση της HTML που χρησιμοποιείται                            \\ \hline
\textless{}head\textgreater{}...\textless{}/head\textgreater{}     & Οι σχετικές πληροφορίες με το έγγραφο (όπως η γλώσσα, η κωδικοποίηση και τα μεταδεδομένα) \\ \hline
\textless{}title\textgreater{}...\textless{}/title\textgreater{}   & Ο τίτλος του αρχείου                                                                      \\ \hline
\textless{}body\textgreater{}...\textless{}/body\textgreater{}     & Τα οπτικά στοιχεία του αρχείου                                                            \\ \hline
\textless{}div\textgreater{}...\textless{}/div\textgreater{}       & Ομαδοποίηση στοιχείων εντός ετικέτας                                                      \\ \hline
\textless{}input\textgreater{}...\textless{}/input\textgreater{}   & Ορισμός πεδίου εισαγωγής                                                                  \\ \hline
\textless{}!--...--\textgreater{}                                  & Ορισμός σχολίων                                                                           \\ \hline
\textless{}form\textgreater{}...\textless{}/form\textgreater{}     & Ορισμός φόρμας                                                                            \\ \hline
\textless{}button\textgreater{}...\textless{}/button\textgreater{} & Ορισμός κουμπιού                                                                          \\ \hline
\textless{}a\textgreater{}...\textless{}/a\textgreater{}           & Ορισμός υπερσυνδέσμου                                                                     \\ \hline
\end{tabular}
\end{table}

\section{CSS}
Η CSS αποτελεί μία γλώσσα επικαλυπτόμενων στυλ μορφοποίησης (Cascading Style Sheets) και όπως η HTML δεν θεωρείται καθαρή γλώσσα προγραμματισμού. Συντελεί στον διαχωρισμό των εντολών εμφάνισης από τις εντολές του περιεχομένου της ιστοσελίδας. Χρησιμοποιείται για τη μορφοποίηση οποιασδήποτε ετικέτας HTLM και τη δημιουργία ενός αποτελέσματος πιο όμορφου οπτικά για τον χρήστη. Η σύνταξη μιας εντολής στην CSS αποτελείται από τρία κύρια στοιχεία: το στοιχείο που θα τροποποιηθεί, τις ιδιότητες του στοιχείου που θα επηρεαστούν και η νέα τιμή των ιδιοτήτων. Στον πίνακα \ref{tbl:css_basic_elements} αναφέρεται ένα παράδειγμα του τρόπου σύνταξης μιας εντολής στην γλώσσα CSS.

\begin{table}[h]
\caption{Στοιχεία εντολής CSS}
\label{tbl:css_basic_elements}
\centering
\begin{tabular}{|l|}
\hline
\begin{tabular}[c]{@{}l@{}}p \{\\     font-family: Calibri\\     color: \#1B1811\\     text-align: left;\\ \}\end{tabular} \\ \hline
\end{tabular}
\end{table}

Όπως φαίνεται στον πίνακα \ref{tbl:css_basic_elements}, το στοιχείο p αποτελεί την ετικέτα της παραγράφου στην HTML και ονομάζεται επιλογέας (css selector). Τα font-family, color και text-align είναι οι ιδιότητες της εντολής CSS και δείχνουν τα στοιχεία του επιλογέα τα οποία θα τροποποιηθούν. Ο διαχωρισμός των ιδιοτήτων από τις νέες τιμές που θα λάβουν σηματοδοτείται με την άνω και κάτω τελεία (:), στης οποίας το αριστερό μέρος βρίσκονται οι ιδιότητες και στο δεξί οι νέες τιμές τους.

\section{JavaScript}
Η JavaScript είναι μια γλώσσα προγραμματισμού σεναρίου (scripting language), η οποία εκτελείται από τους περιηγητές ιστών (web browsers) χρησιμοποιώντας έναν σχετικό διερμηνευτή (Interpreter). Στόχο της αποτελεί η βελτίωση της εμπειρίας χρήσης. Με την JavaScript υπάρχει η δυνατότητα προσθήκης ή αφαίρεσης HTML στοιχείων και CSS κανόνων καθώς και η τροποποίηση ιδιοτήτων HTLM στοιχείων με την μεταβολή στις τιμές τους. Αν και η JavaScript αποτελεί μια γλώσσα προγραμματισμού με το Client-Side χαρακτηριστικό, τον τελευταίο καιρό γίνεται χρήση της και από την πλευρά των υπολογιστών εξυπηρετητών.
 
\section{Ajax}
Η Ajax (Asynchronous JavaScript and XML) αποτελεί τον συνδυασμό των τεχνολογιών JavaScript και XML. Είναι μία τεχνολογία, η οποία προσδίδει διαδραστικές δυνατότητες σε μία ιστοσελίδα. Μέσω της Ajax γίνεται εφικτή η ανανέωση μέρους της ιστοσελίδας (στο παρασκήνιο θα γίνει επικοινωνία της τεχνολογίας με τον server, ο οποίος θα λάβει τα δεδομένα που ζητήθηκαν και με τη σειρά του θα τα εμφανίσει στον χρήστη), χωρίς να χρειαστεί να γίνει ανανέωση (refresh) ολόκληρης της ιστοσελίδας.

\section{PHP}
Η PHP είναι μια γλώσσα προγραμματισμού γενικής χρήσης (παλαιότερα αποτελούσε γλώσσα προγραμματισμού σεναρίου), η οποία χρησιμοποιείται κυρίως για την ανάπτυξη διαδικτυακών εφαρμογών (web apps) και προγραμματιδίων συστήματος (system scripts), δηλαδή αποτελεί την κατάλληλη γλώσσα για την δημιουργία ιστοχώρων με δυναμικό περιεχόμενο. Διατίθεται σε δύο μορφές, σε μορφή πηγαίου κώδικα και σε δυαδική μορφή και οι δύο αυτές μορφές έχουν ελεύθερη πρόσβαση. Η PHP χρησιμοποιείται για τον χειρισμό λειτουργιών και εργασιών τις οποίες θα υλοποιήσει και όχι για την οπτική διαμόρφωση μίας ιστοσελίδας. Επομένως, ο χρήστης λαμβάνει τα αποτελέσματα του σεναρίου (στον browser ως απλές σελίδες HTLM) και όχι τον κώδικα, ο οποίος εκτελείται στον server με τη χρήση του αντίστοιχου διερμηνευτή (interpreter) της κάθε γλώσσας. Ο διερμηνευτής, αφού διαβάσει τον κώδικα, εκτελεί τις δηλώσεις της γλώσσας ανά βήμα και τις μετατρέπει σε εκτελέσιμο κώδικα για το υπολογιστικό σύστημα. Αναλυτικότερα, η PHP έχει την ικανότητα να δημιουργήσει, γράψει, διαβάσει, ανοίξει, κλείσει και διαγράψει αρχεία στη Βάση Δεδομένων. Υποστηρίζει ένα ευρύ φάσμα Βάσεων Δεδομένων, καθώς επίσης είναι δωρεάν και αρκετά εύκολη στην εκμάθηση, ενώ ταυτόχρονα είναι συμβατή και μπορεί να τρέξει σε οποιαδήποτε πλατφόρμα, όπως για παράδειγμα Windows, Mac OS X, Linux, Unix κ.α. Ένα αρχείο PHP (έγγραφο κειμένου αποθηκευμένο με την κατάληξη .php) είναι δυνατό να περιέχει εκτός από τον κώδικα PHP και κώδικα HTLM, CSS και JavaScript. (image and functions here). Είναι σηματνικό να αναφερθεί ότι η PHP επιτρέπει την χρήση του αντικειμενοστραφούς προγραμματισμού (Object Oriented Programming / OOP), ο οποίος αποτελεί πρότυπο στην ανάπτυξη εφαρμογών, σε οποιαδήποτε γλώσσα. Ο αντικειμενοστραφής προγραμματισμός είναι ένα μοντέλο προγραμματισμού που βοηθά στην οργάνωση του σχεδιασμού λογισμικού γύρω από δεδομένα ή αντικείμενα. Το μοντέλο αυτό συμβάλλει στη βελτίωση της σαφήνειας και της οργάνωσης του κώδικα, καθιστώντας τον ευκολότερο στην κατανόηση και τη συντήρησή του. Ένα αντικείμενο μπορεί να οριστεί ως ένα πεδίο δεδομένων που έχει μοναδικά χαρακτηριστικά και συμπεριφορά.
