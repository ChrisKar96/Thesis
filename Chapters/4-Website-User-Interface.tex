\chapter{Διεπαφή Χρήστη}
Σε αυτό το κεφάλαιο, αναλύονται οι λειτουργίες που παρέχει η διαδικτυακή εφαρμογή στους χρήστες της μέσω User Interface (Διεπαφή Χρήστη). Ανάλογα με τον ρόλο του χρήστη που συνδέεται στην εφαρμογή, παρέχονται και οι αντίστοιχες λειτουργίες. Η εμφάνιση της εφαρμογής έχει ίδια αισθητικά χαρακτηριστικά για κάθε είδος χρήστη.Το User Interface θεωρείται από τα πιο σημαντικά μέρη μιας διαδικτυακής εφαρμογής καθώς, αν είναι σχεδιασμένο σωστά, ώστε να είναι εύχρηστο και διαισθητικό, κάνει εύκολη τη χρήση των λειτουργιών του συστήματος, ακόμη και σε χρήστες που πιθανώς δεν είναι εξειδικευμένοι στον τομέα της πληροφορικής.

Για την κατασκευή του User Interface, χρησιμοποιήθηκε η αρχή του Reactive Programming, δηλαδή της μεθόδου προγραμματισμού ιστού ώστε το περιεχόμενο της εφαρμογής να ανανεώνεται δυναμικά και ασύγχρονα με τις ενέργειες του χρήστη, σε ένα εν γένει στατικό περιβάλλον. Όσον αφορά το σχεδιασμό του γραφικού περιβάλλοντος, χρησιμοποιήθηκε Responsive Design, ώστε η εφαρμογή να έχει ομοιόμορφη, καλαίσθητη και λειτουργική εμφάνιση, τόσο σε συσκευές με μεγάλη διάμετρο οθόνης, όπως ηλεκτρονικούς υπολογιστές, όσο και σε συσκευές με μικρότερη διάμετρο οθόνης, όπως κινητά τηλέφωνα.

Ο συνδυασμός των παραπάνω μεθόδων προγραμματισμού ιστού, προσφέρει ευχάριστη εμπειρία χρήσης της εφαρμογής, σε όλους τους χρήστες, ανεξάρτητα από τον τύπο συσκευής που χρησιμοποιούν.

\section{Εγγραφή και Σύνδεση στο Σύστημα}

\section{Μενού Πλοήγησης}

\section{Υπολογιστές}

\section{Ομάδες Υπολογιστών}

\section{Εργαστήρια}

\section{Λειτουργικά Συστήματα}

\section{Διαμορφώσεις}
